\documentclass{article}
\usepackage{packages}

\newcommand{\im}{\mathrm{Im}}

\title{Коллоквиум по Дискретной математике, 2 курс}
\author{Залялов Александр, @bcategorytheory,\\  Солодовников Никита, @applied\_memes \\ Шморгунов Александр, @Owlus }
\date{}

\begin{document}

\maketitle

\setcounter{section}{19}
\section{Дизъюнкты, универсальные дизъюнкты. Исчисление резолюций (ИР) для доказательства несовместности множеств универсальных дизъюнктов. Теорема корректности ИР.}
\begin{definition}
	\textit{Дизъюнктом} называется дизъюнкция атомарных формул и их отрицаний.\\ 
	Пример: $P(x) \vee \bar Q(y, f(x)) \vee s$ (в данном случае $P(x), \bar Q(y, f(x)), s$ --- атомарные формулы, а $\vee$ --- операция дизъюнкции)
\end{definition}
\begin{definition}
	\textit{Универсальным дизъюнктом} называется формула, полученная из дизъюнкта приписыванием кванторов всеобщности. Пример: $\forall x\forall y [P(x) \vee \bar Q(y, f(x)) \vee s]$
\end{definition}
Для того, чтобы доказывать несовместность множеств универсальных дизъюнктов (несовместность конъюнкции всех универсальных дизъюнктов этого множества) мы пользуемся правилами в исчислении резолюций
\vspace{2mm}

Правил в исчислении резолюций два:
\begin{itemize}
	\item $A \vee p,\ B \vee \bar p \rightarrow A \vee B$ (правило резолюций)
	\item $\forall x D(x) \rightarrow D(t)$ для некоторого $t$ 
\end{itemize}

\begin{theorem}{(Теорема корректности исчисления резолюций)}
	Если из набора универсальных дизъюнктов можно вывести пустой дизъюнкт, то этот набор несовместен.
\end{theorem}
\begin{proof}
	Идем методом от противного. Пускай существует модель $M$, в которой все данные дизъюнкы истинны.
	Заметим, что оба правила исчисления резолюций сохраняют истинность.
	
	Действительно, если в правиле резолюций для $A \vee p,\ B \vee \bar p \rightarrow A \vee B$ мы получим, что $A \vee B$ ложно, то это значит, что и $A$, и $B$ ложно, однако если это так, то ложно либо $A \vee p$, либо $B \vee \bar p$
	
	В правиле резолюций для $\forall x D(x) \rightarrow D(t)$, если выражение $D(x)$ истинно для любого $x$, то оно будет истинно для и для некоторого $t$ \footnote{В случае $\forall x \exist y, x < y$ не выполняется, если мы поставим $x = y$. Однако наши универсальные дизъюнкты не допускают квантора существования, поэтому такая формула невозможна}
	
	Если мы вывели пустой дизъюнкт, то по истинности правил исчисления резолюций получаем, что пустой дизъюнкт является истинным. Противоречие.
\end{proof}

\section{Непротиворечивые теории. Теорема полноты ИР (для множеств универсальных дизъюнктов).}
\begin{definition}
	\textit{Непротиворечивой теорией} называется теория такая, что в ней утверждение не может быть одновременно доказано и опровергнуто
\end{definition}

\begin{theorem}{(Теорема полноты исчисления резолюций)}
	Если набор универсальных дизъюнктов несовместен, то из него можно вывести пустой дизъюнкт
\end{theorem}
\begin{proof}
	Пускай есть счётное множество универсальных дизъюнктов $S$. Заметим, что если мы подставим вместо терм конкретные значения в этом множестве, то можем заменить все атомарные формулы пропозициональными переменными (константа 0 исчезает, так как дизъюнкция; константу 1 заменяем на дизъюнкцию переменных $p \vee \bar p$, $p$ в данном случае --- новая переменная, которую мы ввели). Назовём это новое множество из пропозициональных переменных $S'$.
	
	Так как $S$ несовместно, то $S'$ тоже несовместно, так как существует набор терм, при котором из $S$ мы получаем $S'$. Значит, мы свели теорему к случаю для дизъюнктов из пропозициональных переменных. Теорема полноты для такого случая доказывалась в билете 16. А раз для любого набора терм вывести пустой дизъюнкт нельзя, то и для $S$ тоже. 
\end{proof}

\section{Исчисление резолюций для теорий, состоящих из формул общего вида (приведение к предваренной нормальной форме и сколемизация). Доказательства общезначимости с помощью ИР. Выводимость формулы в теории с помощью ИР. Теорема компактности.}
Для того, чтобы применить правила исчислений резолюций для теорий, состоящих из формул общего вида необходимо для начала привести их к форме, состоящей из конъюнкций, дизъюнкций, отрицаний атомарных формул и кванторов всеобщности. Поэтому каждую формулу приводят сначала к \textit{предваренной нормальной форме}, а затем к \textit{сколемовской нормальной форме}

\begin{definition}
	\textit{Нормальной формой} называется формула, состоящая только из конъюнкций, дизъюнкций, отрицаний атомарных формул и кванторов
\end{definition}

\begin{definition}
	\textit{Предваренной нормальной формой} называется нормальная форма формулы, кванторы которой стоят в начале
\end{definition}

\begin{definition}
	\textit{Сколемовской нормальной формой} называется предваренная нормальная форма формулы, кванторы которой являются кванторами всеобщности
\end{definition}

Для приведения к нормальной форме нужно преобразовывать нестандартные операции в вид конъюнкций, дизъюнкций и отрицаний (Пример: $A \rightarrow B \equiv \overline{A} \vee B$), а также пользоваться следующими правилами:
\begin{itemize}
	\item $\overline{\exists x A(x)} \equiv \forall x\overline{A(x)}$
	\item $\overline{A \vee B} \equiv \overline{A} \wedge \overline{B}$
\end{itemize}

Для приведения к предваренной нормальной форме нужно вынести кванторы. Делается это при помощи следующих правил:
\begin{itemize}
	\item $C \vee \forall x A(x) \equiv \forall x[C \vee A(x)]$, $C$ не зависит от $x$, операции между A(x) и C могут быть любые ($\vee$ или $\wedge$), квантор для $x$ тоже любой ($\forall$ или $\exists$)
	\item $\forall x A(x) \vee \forall x B(x) \equiv \forall x\forall y[A(x) \vee B(y)]$, тоже для любых операций и кванторов
	\item $\forall x A(x) \vee \forall x B(x) \vdash \forall x[A(x) \vee B(x)]$, для любых операций
	\item $\exists x[A(x) \vee B(x)] \vdash \exists x A(x) \vee \exists B(x)$, для любых операций
\end{itemize}
	
Для приведения к сколемовской нормальной форме будем использовать следующую операцию:
\begin{itemize}
	\item Убираем самый левый квантор существования. Заменяем его в атомарных формулах на функцию от всех предыдущих кванторов всеобщности.
	
	Пример: $\forall x \exists y \forall z \exists w[A(x, w) \vee B(y, z)] \vdash \forall x \forall z \exists w[A(x, w) \vee B(f_y(x), z)] \vdash \forall x \forall z[A(x, f_w(x, z)) \vee B(f_y(x), z)]$
	
	Функции $f_y, f_w$ называются \textit{сколемовскими функциями}
\end{itemize}

Доказательство общезначимости с помощью ИР и выводимость формулы в теории с помощью ИР (напомнить @Owlus для завершения)

Пусть $T$ --- множество формул (может быть как конечным, так и счётным)
\begin{theorem}{(Теорема компактности)}
	Если $T$ несовместно, то у него существует несовместное конечное подмножество $T'$
\end{theorem}

\begin{proof}
    Для конечного множества очевидно (берём в качестве подмножества само множество)
	
	Для счётного множества: так как $T$ несовместно, то из него можно вывести пустой дизъюнкт (теорема полноты ИР, билет 21). Причём этот пустой дизъюнкт выводится из конечного числа исходных формул (так как сами формулы конечные и число операций конечно). Поэтому возьмём в качестве $T'$ эти формулы. Из них выводится пустой дизъюнкт (теорема корректности ИР, билет 20), а значит $T'$ несовместна
\end{proof}

\section{Гомоморфизмы, эпиморфизмы (сюръективные гомоморфизмы), изоморфизмы. Теорема о сохранении истинности при эпиморфизме. Изоморфные модели. Элементарно эквивалентные модели, элементарная эквивалентность изоморфных моделей.}
$\Gamma$ --- сигнатура

$M_1, M_2$ --- интерпретации $\Gamma$
\begin{definition}
	\textit{Гомоморфизмом $h: M_1 \rightarrow M_2$} называется отображение, сохраняющее все предикаты и формулы в сигнатуре. 
	
	Для предиката $P^n \in \Gamma$, его интерпретаций $P_1 \in M_1, P_2 \in M_2$ действует \\
	$\forall a_1, a_2,..., a_n[P_1(a_1, a_2,..., a_n) = P_2(h(a_1), h(a_2),..., h(a_n))]$
	
	Для формулы $f^n \in \Gamma$, его интерпретаций $f_1 \in M_1, f_2 \in M_2$ действует \\
	$\forall a_1, a_2,..., a_n[h(f_1(a_1, a_2,..., a_n)) = f_2(h(a_1), h(a_2),..., h(a_n))]$
\end{definition}

\begin{definition}
	\textit{Эпиморфизмом (сюръективным гомоморфизмом) $h: M_1 \rightarrow M_2$} называется, если $h$ - гомоморфизм и является сюръекцией
\end{definition}

\begin{definition}
	\textit{Изоморфизмом $h: M_1 \rightarrow M_2$} называется, если $h$ - гомоморфизм и является биекцией
\end{definition}

\begin{definition}
	\textit{$M_1$ элементарно эквивалентно $M_2$ ($M_1 \cong M_2$)}, если для любой замкнутой формулы $A$ сигнатуры $\Gamma$ $M_1 \vDash A \Leftrightarrow M_2 \vDash A$ (формула истинна в $M_1$ т. и т.т, когда она истинна в $M_2$)
\end{definition}

Пусть $h: M_1 \rightarrow M_2$ --- эпиморфизм

$A(x_1,..., x_n)$ --- произвольная формула
\begin{theorem}{(Теорема о сохранении истинности при эпиморфизме)}
	Для всех элементов $a_1,..., a_n \in M_1, M_1 \vDash A(a_1,..., a_n) \rightleftarrow M_2 \vDash A(h(a_1),..., h(a_2))$ 
\end{theorem}

\begin{proof}
    \begin{lemma}
        для терма $t$ верно, что $h(|t(a_1,..., a_n)|_1) = |t(h(a_1),..., h(a_n))|_2$ ($|t(...)|_1$ - терм в интерпретации $M_1$)
    \end{lemma}
    
    \begin{proof}
        Пусть $t = f(x, g(y))$. Тогда $|t(a,b)|_1 = f_1(a, g_1(b))$
        
        $h(f_1(a, g_1(b)) = f_2(h(a), h(g_1(b))) = f_2(h(a), g_2(h(b)))$, мы пользуемся тем, что при гомоморфизме функции и предикаты сохраняются
    \end{proof}
    
    Пользуемся идукцией по построению $A$
    \begin{enumerate}
        \item База индукции: $A = P(t_1,..., t_m)$
        
        $M_1 \vDash A(a_1,...,a_m) \leftrightarrow P_1(|t_1(a_1,..., a_m)|_1,...) = 1 \Leftrightarrow P_2(h(t_1(a_1,..., a_m)),...) = 1$\\
        $\Leftrightarrow P_2(|t_1(h(a_1),..., h(a_m))|_2)$ (по лемме) $\leftrightarrow M_2 \vDash A(h(a_1),..., h(a_m))$
        
        \item Индуктивный переход: разбиваем на случаи:
        \begin{itemize}
            \item $A = B \vee C$
            
            $M_1 \vDash (B(a,...) \vee C(a,...)) \Leftrightarrow M_1 \vDash B$ или $M_1 \vDash C \Leftrightarrow M_2 \vDash B(h(a),...)$ или $M_2 \vDash C(h(a),...) \Leftrightarrow M_2 \vDash (B \vee C)(h(a),...)$ (аналогично доказывается для остальных операций)
            \item $A = \exists x B(x, y)$
            
            $M_1 \vDash \exists x B(x, a) \leftrightarrow \exists b \in M_1, M_1 \vDash B(b, a) \leftrightarrow \exists b \in M_1, M_2 \vDash B(h(b), h(a)) \leftrightarrow \exists c \in M_2, M_2 \vDash B(c, h(a)) \leftrightarrow M_2 \vDash \exists x B(x, h(a))$ (предпоследняя эквивалентность верна именно потому, что у нас эпиморфизм), квантор всеобщности доказывается аналогично 
        \end{itemize}
    \end{enumerate}
\end{proof}

\begin{theorem}{(Элементарная эквивалентность изоморфных моделей)}
	Если модели $M_1$ и $M_2$ изоморфны, то они элементарно эквивалентны \footnote{Для элементарной эквивалентности достаточно даже просто эпиморфизма, но мы используем более частую формулировку теоремы}
\end{theorem}

\begin{proof}
    Рассматриваем только замкнутые формулы $A$. Из предыдущей теоремы следует, что формула $A$ схраняет свою истинность при изоморфизме, а это значит, что $M_1$ и $M_2$ будут элементарно эквивалентны по определению.
\end{proof}


\end{document}