\documentclass{article}
\usepackage{packages}

\title{Коллоквиум по Дискретной математике, 2 курс}
\author{Залялов Александр, @bcategorytheory,\\  Солодовников Никита, @applied\_memes \\ Шморгунов Александр, @Owlus }
\date{}

\setcounter{section}{26}
\section{Игры Эренфойхта} 

Цель: сформулировать общий критерий
элементарной эквивалентности двух интерпретаций некоторой сигнатуры (считаем, что сигнатура содержит только предикатные символы).



Критерий будет сформулирован в терминах некоторой игры, называемой игрой Эренфойхта. В ней участвуют два игрока, называемые Новатором (Н) и Консерватором (К). Игра определяется выбранной парой интерпретаций.

В начале игры Новатор объявляет натуральное число $k$. Далее
они ходят по очереди, начиная с Н; каждый из игроков делает $k$ ходов, после чего определяется победитель.

На $i$-м ходу Н выбирает элемент в одной из интерпретаций (в
любой из двух) и помечает его числом i. В ответ К выбирает некоторый элемент из другой интерпретации и также помечает его числом i.

После k ходов игра заканчивается. При этом в каждой интерпретации k элементов оказываются помеченными числами от $1$ до $k$ (мы не учитываем, кто именно из игроков их пометил). Обозначим эти элементы
$a_1$, $a_2$, $\cdots$ , $a_k$ (для первой интерпретации)
и $b_1$, $b_2$, $\cdots$ , $b_k$ (для второй). Элементы $a_i$ и $b_i$ (с одним и тем же i) будем называть соответствующими друг другу.

Посмотрим, найдётся ли предикат сигнатуры, который различает помеченные элементы первой и второй интерпретации (то есть истинен на некотором наборе помеченных элементов в одной интерпретации, но ложен на соответствующих элементах другой). Если такой предикат найдётся, то выигрывает Новатор, в противном случае — Консерватор.

\begin{theorem}
Интерпретации элементарно эквивалентны $\Longleftrightarrow$ К имеет выигрышную стратегию в этой игре.
\end{theorem}

\begin{proof}
Число ходов, которое понадобится Новатору, соответствует кванторной глубине различающей интерпретации формулы. Кванторная глубина формулы определяется так:
\begin{itemize}
    \item Глубина атомарных формул равна нулю.
    \item Глубина формул $\phi \lor \psi$ и ϕ∧ψ равна максимуму глубин формул $\phi$ и $\psi$.
    \item Глубина формулы $\neg \phi$ равна глубине формулы $\phi$.
    \item Глубина формул $\exists \xi$ $\phi$ и $\forall \xi$  $\phi$ на единицу больше глубины формулы $\phi$.
\end{itemize}

Рассмотрим позицию, которая складывается в игре после $k$ ходов Н и К (перед очередным ходом Н) и за $l$ ходов до конца игры.

\begin{lemma} Если есть формула глубины $l$ с параметрами $x_1$, $\cdots$ , $x_k$, отличающая $a_1$, $\cdots$ , $a_k$ от $b_1$, \cdots , $b_k$, то в указанной позиции Н имеет
выигрышную стратегию; в противном случае её имеет К.
\end{lemma}

\begin{proof}

Пусть такая формула $\phi$ существует. Она представляет собой
бескванторную комбинацию некоторых формул вида
$\forall \xi$ $\psi$ и $\exists \xi$ $\psi$, где $\psi$ — формула глубины меньше $l$. Хотя бы одна из этих формул должна также отличать $a_1$, $\cdots$ , $a_k$ от $b_1$, \cdots , $b_k$ (ереходя к отрицанию, можно считать, что эта формула
начинается с квантора существования).

Пусть формула $\phi$, имеющая вид
$\exists x_{k+1} \psi(x_1, . . . , x_k, x_{k+1})$,
истинна для $a_1$, $\cdots$ , $a_k$ и ложна для $b_1$, \cdots , $b_k$. Тогда найдётся такое $a_{k+1}$, для которого в A истинно $\psi(a_1, \cdots , a_k)$.
Это $a_{k+1}$ и будет выигрывающим ходом Н; при любом ответном ходе $b_{k+1}$ К формула $\psi(b_1, \cdots , b_k, b_{k+1})$
будет ложной. Таким образом, некоторая формула глубины l - 1
отличает наборы A и B. 

Рассуждая по индукции,
мы можем считать, что в оставшейся (l - 1)-ходовой игре Н имеет выигрышную стратегию. (В конце концов мы придём к ситуации,
когда некоторая бескванторная формула отличает $k + l$ элементов в A от соответствующих элементов в B, то есть Н выиграет.)

Пусть такой формулы нет.

Будем называть две формулы эквивалентными, если они одновременно истинны или ложны в любой интерпретации на любом наборе аргументов. Поскольку сигнатура конечна,
существует лишь конечное число атомарных формул, все параметры которых содержатся среди $u_1$, $\cdots$ , $u_s$. Существует лишь конечное число булевых функций с данным набором аргументов, поэтому существует лишь конечное число неэквивалентных бескванторных формул, все параметры которых содержатся среди $u_1$, $\cdots$ , $u_s$. Отсюда следует, что существует лишь конечное число неэквивалентных
формул вида $\exists u_s \psi(u_1, \cdots , u_s)$,
и потому лишь конечное число неэквивалентных формул глубины 1,
параметры которых содержатся среди $u_1, \cdots , u_{s-1}$. Продолжая эти рассуждения, мы заключаем, что для любого $l$ и для любого набора переменных
$u_1, \cdots , u_n$ существует лишь конечное число неэквивалентных формул глубины $l$, все параметры которых содержатся среди $u_1, \cdots , u_n$.


Вернёмся к игре Эренфойхта. Пусть элементы $a_1, \cdots , a_k$ нельзя отличить от элементов $b_1, \cdots , b_k$ с помощью формул глубины $l$. Пусть Н выбрал произвольный
элемент в одной из интерпретаций, скажем, $a_{k+1}$. Рассмотрим все
формулы глубины l - 1 с k + 1 параметрами (с точностью до эквивалентности их конечное число); некоторые из них будут истинны на $a_1, \cdots , a_{k+1}$, а некоторые ложны. Тогда формула, утверждающая существование $a_{k+1}$ с ровно такими свойствами (после квантора существования идёт конъюнкция всех истинных формул и отрицаний всех ложных) будет формулой глубины l, истинной на $a_1, \cdots , a_{k}$. По
предположению эта формула должна быть истинной и на $b_1, \cdots , b_{k}$, и потому существует $b_{k+1}$ с теми же свойствами, что и $a_{k+1}$. Этот
элемент $b_{k+1}$ и должен пометить К. Теперь предположение индукции позволяет заключить, что в возникшей позиции у К есть выигрышная стратегия.
\end{proof}
Лемма доказана. Её частным случаем является обещанный критерий элементарной эквивалентности

\end{proof}

\setcounter{section}{27}
\section{Семантически полные теории. Критерий семантической полноты теории в терминах эле-
ментарной эквивалентности моделей. Аксиоматизация элементарной теории упорядоченного
множества целых чисел.}

\begin{definition}\textit{Аксиоматическая теория T} - множество замкнутых формул.
\end{definition}

T \textit{семантически полна}, если для любой замкнутой формулы А выполнено одно из двух:
\begin{enumerate}
    \item из T семантически следует А (А истинно во всех моделях теории)
    \item из Т семантически следует $\neg A$
\end{enumerate}

\begin{lemma}
Теория семантически полна $\Longleftrightarrow$ любые 2 ее модели элементарно эквивалентны.
\end{lemma}
\begin{proof}

$\Rightarrow$ Элементарная эквивалентность значит, что в обоих моделях любая формула или истинна, или ложна. Тогда если $\phi$ следует из А, то она истинна для всех моделей, следовательно, для каждой пары. Аналогично для $\neg \phi$

$\Leftarrow$ От противного: какая-то формула сама не следует и ее отрицание не следует. Значит, есть модели, в одной из которых А истинно, в другой - ложно. Противоречие с элементарной эквивалентностью.

\end{proof}

\noindent\textbf{Аксиоматизация множества рациональных чисел}
\vspace{2mm}

$M = (Q, =, <)$
\begin{itemize}
    \item аксиомы равенства
    \begin{enumerate}
        \item $\forall x \: x = x$
        \item $\forall x \forall y \: x = y \rightarrow y = x$
        \item $\forall x \forall y \forall z \: \: x = y \land y = z \rightarrow x = z$
        \item $\forall x_1 \forall x_2 \forall y_1 \forall y_2 \: \: x_1 = x_2 \land y_1 = y_2 \rightarrow (x_1 = x_2 \rightarrow y_1 = y_2)$
    \end{enumerate}
    \item аксиомы линейного порядка
    \begin{enumerate}
        \item $x < y \land y < z \rightarrow x < z$
        \item $\neg \: (x < x)$
        \item $\forall x \forall y \: x < y \lor x > y \lor x = y$
    \end{enumerate}
    \item отсутствие наибольшего и наименьшего элемента
    \item плотность множества $\forall x, y \: (x < y \rightarrow \exists z \: \:  x < z \land z < y)$
\end{itemize}

\begin{theorem}{T - совместная и семантически полная.}
\begin{proof}
Доказательство аналогично игре Эренфойхта с R и Q. Все выбранные в одной модели элементы идут в том же порядке, что и элементы второй модели. Консерватору достаточно возможности выбрать элемент между любыми двумя и отсутствие наибольшего и наименьшего элемента. 
\end{proof}
\end{theorem}

\setcounter{section}{28}
\section{Аксиоматизация множества целых чисел.}

$M = (Z, =, <)$
\begin{itemize}
    \item аксиомы равенства
    \item аксиомы линейного порядка
    \item отсутствие наибольшего и наименьшего элемента
    \item $\forall x \exists y (x < y \land \neg (\exists z \: \: x < z \land z < y))$
    \item $\forall x \exists y (x > y \land \neg (\exists z \: \: x > z \land z > y))$
\end{itemize}

\theorem{T - совместная и семантически полная.}
\begin{proof}
Как устроены модели Т? Это Z, Z+Z или любое множество вида AZ (А - линейно упорядоченное множество, в каждом элементе которого лежит множество целых чисел). 
Скажем, что элементы эквивалентны, если мы можем получить один из другого за конечное число шагов. Факторизуем по этому отношению эквивалентности.
\end{proof}

\begin{lemma}Для любого линейно упорядоченного $A \: \: AZ \cong Z$
\begin{proof}
Доказывается аналогично случаю с Z+Z
\end{proof}
\end{lemma}
\end{theorem}
