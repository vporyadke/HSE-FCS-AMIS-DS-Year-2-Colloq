\documentclass{article}
\usepackage{packages}

\newcommand{\im}{\mathrm{Im}}

\title{Коллоквиум по Дискретной математике, 2 курс}
\author{Залялов Александр, @bcategorytheory,\\  Солодовников Никита, @applied\_memes \\ Шморгунов Александр, @Owlus }
\date{}

\begin{document}

\maketitle

\setcounter{section}{9}
\section{Определение машин Тьюринга и вычислимых на машинах Тьюринга функций. Тезис Чёрча-Тьюринга. Неразрешимость проблемы остановки машины Тьюринга.}
Машина Тьюринга задаётся\footnote{Здесь машина Тьюринга определяется в соответсвии с лекцией. Следует понимать, что это определение не является общепринятым. Вариаций масса: кто-то запрещает головке оставаться на месте, кто-то выделяет выходной алфавит, отличный от входного и т. д.}
\begin{itemize}
	\item непустым конечным алфавитом $\Sigma$, среди которого выделен пробельный символ $\vartextvisiblespace$ и не содержащее пробела подмножество $\Gamma$ --- входной алфавит;
	\item непустым конечным множеством состояний $Q$, среди которых выделено начальное состояние $s_0$ и множество терминальных состояний $F$;
	\item функцией переходов $\delta:(Q \setminus F) \times \Sigma \to Q \times \Sigma \times \{-1, 0, +1\}$.
\end{itemize}

Машина Тьюринга состоит из бесконечной ленты, разбитой на ячейки, головки, в любой момент времени указывающей на одну ячейку и одной ячейки памяти, в которой хранится текущее состояние. В начальный момент времени на ленте записано некоторое слово, составленное из букв входного алфавита, головка смотрит на первый символ этого слова, во всех остальных ячейках пробелы. Затем в каждый момент времени вычисляется $\delta(q, c) = (q', c', \Delta)$, где $q$ --- текущее состояние, $c$ --- символ записанный в ячейке, на которую сейчас смотрит головка. Состояние меняется на $q'$, символ в текущей ячейке на $c'$, головка остаётся на месте или передвигается на один влево или вправо в соответствии со значением $\Delta$. Если $q'$ оказалось терминальным, на этом работа машины заканчивается, иначе этот процесс продолжается.

Машины Тьюринга естественным образом отождествляются с частичными функциями $f:\Gamma^* \to \Gamma^*$ --- аргументом функции является входное слово, а возвращает функция слово, записанное на ленте после завершения работы машины(то есть всё, что написано на ленте, кроме бесконечного числа пробелов слева и справа). Функции будут частичными, поскольку машина Тьюринга может продолжать работать бесконечно или в данной конструкции может оказаться, что на выходе есть символ, не содержащийся в $\Gamma$. Функции, которые можно таким образом получить по некоторой машине Тьюринга, называются вычислимыми на машине Тьюринга.
\paragraph{Тезис Чёрча-Тьюринга.} \textit{Любая вычислимая функция вычислима на машине Тьюринга.} 

Здесь понятие "вычислимая функция" используется в неформальном смысле, под ним понимается функция, вычислимая в любой разумной модели, которая может прийти вам в голову. Тезис не является формальным утверждением, он никак не доказывается и принимается нами на веру.

\begin{theorem}
	Не существует вычислимой функции, определяющей по машине Тьюринга и входному слову, остановится ли эта машина. 
\end{theorem}
Теперь, когда мы отождествили вычислимые и вычислимые на машине Тьюринга функции, эта теорема непосредственно следует из доказательства теоремы о существовании полного перечислимого множества из 7 билета.

\section{Неразрешимость проблемы достижимости в односторонних ассоциативных исчислениях. Полугруппы, заданные порождающими и соотношениями. Теорема Маркова–Поста: неразрешимость проблемы равенства слов в некоторой конечно определенной полугруппе (без доказательства).}

\begin{definition}
	\textit{Односторонним ассоциативным исчислением} называется множество из всех слов над некоторым конечным алфавитом и конечный набор подстановок. Каждая подстановка представляет собой пару слов $(s, t)$ и позволяет в любом слове содержащем $s$ как подстроку заменить её на $t$ (но не наоборот).
\end{definition}

\begin{theorem}
	Существует одностороннее ассоциатвиное исчисление, в котором не разрешима задача проверить по паре слов, можно ли некоторой последовательностью подстановок перейти от первого ко второму.
\end{theorem}

\begin{proof}
	Возьмём некоторую машину Тьюринга $M$, для которой неразрешима проблема остановки, при чём если такую, что если она останавливается, то на ленте записано пустое слово. Построим по ней одностороннее ассоциативное исчисление, в котором из $[X]$ можно получить $Y$, если и только если $M$ преобразует $X$ в $Y$. В качестве алфавита для исчисления возьмём объединение алфавита $M$ и её множества состояний (а также квадратные скобки и символы $\triangleleft, \triangleright$). Будем сопоставлять конфигурациям машины слова исчисления. Если машина находится в состоянии $s$, на ленте записано слово $PQ$(конкатенация слов $P$ и $Q$) и головка указывает на первый символ слова $Q$, сопоставим такой конфигурации слово $[PsQ]$ в нашем исчислении. Тут важно, что мы считаем, что у машины не пересекаются алфавит и множество состояний. Построим по переходам машины подстановки для исчисления.
	
	\begin{center} \begin{tabular}{c | c }
		Переход МТ & Подстановка одностороннего ассоциативного исчисления \\ \hline
		$(s, c) \mapsto (s', c', 0)$ & $sc \to s'c'$ \\
		$(s, c) \mapsto (s', c', +1)$ & $sc \to c's'$ \\
		$(s, c) \mapsto (s', c', -1)$ & $xsc \to s'xc'$ --- для каждого символа $x$ из алфавита машины, а также $[sc \to [s'\vartextvisiblespace c'$ \\
		$(s, \vartextvisiblespace) \mapsto (s', c', 0)$ & $s] \to s'c']$ \\
		$(s, \vartextvisiblespace) \mapsto (s', c', +1)$ & $s] \to c's']$ \\
		$(s, \vartextvisiblespace) \mapsto (s', c', -1)$ & $xs] \to s'xc']$
	\end{tabular} \end{center}

	Дополнительно к этому введём подстановки, позволяющие получить пустое слово, если машина остановится.
	\begin{itemize}
		\item $f \to \triangleleft$, $f$ --- терминальное состояние;
		\item $c\triangleleft \to \triangleleft, c \ne [$;
		\item $[\triangleleft \to \triangleright$;
		\item $\triangleright c \to \triangleright, c \ne ]$;
		\item $\triangleright ] \to \varepsilon$(пустое слово).
	\end{itemize}

	Это можно было бы реализовать проще без двух дополнительных символов, но так мы получаем, что всегда существует ровно одна последовательностей подстановок, моделирующих работу машины Тьюринга. Осталась одна деталь --- мы пообщеали, что мы начнём с $[X]$, а не с $[s_0X]$. Она решается просто --- добавлением подстановки $[x \to [s_0x$ для всех символом $x$ из алфавита машины. 

	Итак, мы свели задачу остановки машины Тьюринга (про которую было известно, что она неразрешима) к задаче достижимости в одностороннем ассоциативном исчислении и показали этим, что эта задача тоже неразрешима.
\end{proof}
Оказывается, если потребовать, чтобы все подстановки были двухсторонними, то задача останется неразрешимой, но доказывать этот факт от нас не требуют. При чём такую задачу можно сформулировать на языке алгебры:

Пусть про некоторую полугруппу известно, что она содержит элементы $a_1, \ldots, a_n$ и в ней выполняются некоторые (конечное количество) равенства вида $a_{i_1}a_{i_2}\ldots a_{i_k} = a_{j_1}a_{j_2}\ldots a_{j_m}$. Обязательно ли в ней выполняется заданное равенство такого же вида?

\section{Исчисление высказываний (аксиомы и правила вывода), понятие вывода. Теорема корректности исчисления высказываний}
Высказываниями мы называем утверждения, которые либо истинны, либо ложны. При этом если $A, B$ являются высказываниями, то $\lnot A, A \lor B, A \land B, A \to B$ --- тоже высказывания. Из такого определения никак не следует, что высказывания вообще существуют, так что в любом применении исчисления высказываний также описывают некоторые атомарные высказывания. Но нам для доказательства общих фактов это никак не потребуется. Исчисление высказываний задаётся аксиомами и правилами вывода. У нас имеется 11 аксиом:
\begin{enumerate}
	\item $(A \to (B \to C)) \to ((A \to B) \to (A \to C))$
	\item $A \to (B \to A)$
	\item $A \land B \to A$
	\item $A \land B \to B$
	\item $A \to (B \to A \land B)$
	\item $A \to A \lor B$
	\item $B \to A \lor B$
	\item $(A \to C) \to ((B \to C) \to ((A \lor B) \to C))$
	\item $\lnot A \to (A \to B)$
	\item $(A \to B) \to ((A \to \lnot B) \to \lnot A)$
	\item $A \lor \lnot A$
\end{enumerate}
и 1 правило вывода(modus ponens)
\[\tfrac{A \to B, \; A}{B} \]

Вывод в исчислении высказываний --- это последовательность из операций двух видов
\begin{itemize}
	\item Подстановка в некоторую аксиому любых высказываний вместо $A, B, C$.
	\item Применение правила вывода. Если уже выведены $A\to B$ и $A$, можно вывести $B$
\end{itemize}

\begin{theorem}{(Корректность исчисления резолюций)}
	Любая формула, которую можно вывести в исчислении высказываний, истинна(тавтологична).
\end{theorem}
Здесь мы называем формулу истинной(тавтологичной), если она как булева формула верна при всех значениях входящих в неё переменных. Отметим, что только в этом контексте мы понимаем $\lnot, \lor, \land$ как привычные логические операции. С точки зрения исчисления высказываний, это просто какие-то символы, всё что мы про них знаем --- это аксиомы и правило.
\begin{proof}
	Достаточно убедиться, что все действия, который мы можем производить в ходе вывода не позволяют получить ложное выражение. Во-первых, все аксиомы истинны при любых значениях входящих в них переменных. Во-вторых, если $A \to B$ истинно и $A$ истинно, то $B$ истинно.
\end{proof}

\section{Вывод из гипотез. Лемма о дедукции. Полезные производные правила.}
Пусть $\Gamma$ --- некоторое множество высказываний(гипотез). Тогда говорят, что формула $A$ выводится из $\Gamma$, если её можно вывести,  разрешая пользоваться не только аксиомами и правилом вывода, но и высказываниями из $\Gamma$. Можно сказать, что у нас появилась третья операция: бесплатно получить формулу из $\Gamma$. Обозначение: $\Gamma \vdash A$. В таких терминах можно сказать, что формула, которую можно вывести в исчислении высказываний,  выводится из пустого множества гипотез, обозначение: $\ \vdash A$.

\begin{lemma}
	$\ \vdash A \to A$
\end{lemma}

\begin{proof}
	\begin{enumerate}
		\item $A \to (A \to A)$(2 аксиома)
		\item $\lnot A \to (A \to A)$ (9 аксиома)
		\item $A \lor \lnot A$ (11 аксиома)
		\item $(A \to (A \to A)) \to ((\lnot A \to (A \to A)) \to ((A \lor \lnot A) \to (A \to A)))$ (8 аксиома, подставлены $A, \lnot A, A \to A$)
		\item $(\lnot A \to (A \to A)) \to (A \lor \lnot A) \to (A \to A)$ (modus ponens)
		\item $(A \lor \lnot A) \to (A \to A)$ (modus ponens)
		\item $A \to A$ (modus ponens)
	\end{enumerate}
\end{proof}

\begin{theorem}{(Лемма о дедукции)}
	$\Gamma \cup \{A\} \vdash B \implies \Gamma \vdash (A \to B)$
\end{theorem}

\begin{proof}
	Пусть с набором гипотез $\Gamma \cup \{A\}$ мы могли вывести формулу $B$, последовательно выводя формулы $B_1, B_2, \ldots, B_n, B_n = B$. По индукции докажем, что с набором гипотез $\Gamma$ можно доказать последовательность $A \to B_1, \ldots, A \to B_n$. Разберём для этого все способы, которыми мы умеем выводить
	\begin{enumerate}
	\item $B_i$ получено как гипотеза. Если $B_i = A$, то по лемме мы сможем вывести $A \to A$. Иначе нам доступен такой вывод: $B_i, B_i \to (A \to B_i), A \to B_i$.
	\item $B_i$ получено подставлением формул в аксиому. Работает последовательность из предыдущего пункта.
	\item $B_i$ получено по modus ponens из $B_j$ и $B_k(j < i , k < i)$. Тогда без потери общности считаем, что $B_j = B_k \to B_i$. По предположению индукции мы уже вывели $A \to B_k$ и $A \to (B_k \to B_i)$. По первой аксиоме выведем $A \to (B_k \to B_i) \to ((A \to B_k) \to (A \to B_i))$. Дважды применив к этому modus ponens, получим $A \to B_i$.
	\end{enumerate}
\end{proof}
Некоторые производные правила --- следствия из леммы о дедукции:
\begin{itemize}
	\item Из $A$ и $B$ можно вывести $A \land B$.
	\item Если из $\Gamma \cup \{A\}$ можно вывести $B$ и $\lnot B$, то из $\Gamma$ можно вывести $\lnot A$ (производное правило доказательства от противного).
	\item Если из $\Gamma \cup \{A\}$ можно вывести $B$ и из $\Gamma \cup \{\lnot A\}$ можно вывести $B$, то из $\Gamma$ можно вывести $B$ (производное правило разбора случаев).
	\item Из $A, \lnot A$ можно вывести что угодно.
\end{itemize}

\section{Теорема полноты исчисления высказываний.}
\begin{lemma}
	Пусть формула $A$ зависит от переменных $p_1, \ldots, p_n$. При этом при $(p_1, \ldots, p_n) = (\varepsilon_1, \ldots, \varepsilon_n)$ формула выдаёт значение $\varepsilon$. Тогда $\{p_1^{\varepsilon_1}, \ldots, p_n^{\varepsilon_n}\} \vdash A^{\varepsilon}$, где
\[P^\varepsilon = \begin{dcases*} \lnot P, & $\varepsilon = 0$ \\ P& $\varepsilon = 1$ \end{dcases*} \]
\end{lemma}

\begin{proof}
	Доказательство индукцией по построению формулы $A$. Разбираем все способы, которыми она была построена, а для них все значения её составных частей.
	\begin{enumerate}
		\item $A = p$. Очевидно, $p \vdash p$ и $\lnot p \vdash \lnot p$.
		\item $A = B \land C$. 
		\begin{enumerate}
			\item Пусть $B$ и $C$ истинны. Тогда по предположению индукции мы можем вывести $B$ и $C$ и требуется показать, что мы можем вывести $B \land C$. Мы умеем это делать по производному правилу.
			\item Пусть $B$ истинно, а $C$ ложно. Тогда по предположению индукции мы можем вывести $B$ и $\lnot C$, и хотим вывести $\lnot(B \land C)$. Воспользуемся правилом доказательства от противного и добавим себе в гипотезы $B \land C$. Из $B \land C$ нетрудно вывести $C$, и мы умеем выводить $\not C$, противоречие достигнуто.
		\end{enumerate}
		Оставшиеся два случая аналогичны второму.
		\item $A = B \lor C$. Если хотя бы одно из $B$ или $C$ истинно, то ясно, что можно вывести $B \lor C$. Пусть теперь мы умеем выводить $\lnot B, \lnot C$ и нужно вывести $\lnot (B \lor C)$. Снова будем выводить от противного и предположим $B \lor C$. Заметим, что из $\lnot B, \lnot C, B$ можно вывести всё что угодно, и из $\lnot B, \lnot C, C$ можно вывести всё что угодно. Тогда всё что угодно можно вывести и из $\lnot B, \lnot C, B \lor C$, в том числе и противоречие.
		\item $A = B \to C$. Для случаев с истинным $B$ или ложным $C$ вывод простой --- достаточно воспользоваться 2 или 9 аксиомой (и modus ponens). Пусть мы умеем выводить $B, \lnot C$ и нужно вывести $\lnot(B \to C)$. Опять докажем от противного и по modus ponens из $B, B \to C$ выведем $C$. Это даст противоречие, поскольку у нас есть $\lnot C$.
		\item $A = \lnot B$. Ясно, что $\lnot B \vdash \lnot B$. Нужно доказать, что $B \vdash \lnot \lnot B$. Для этого нужно в очередной раз доказать от противного и вывести из $B, \lnot B$ какое-нибудь противоречие. Но $B, \lnot B$ уже противоречие.
	\end{enumerate}
\end{proof}

\begin{theorem}{(Полнота исчисления высказываний)}
	Любую тавтологию можно вывести в исчислении высказываний.
\end{theorem}

\begin{proof}
	Пусть тавтология $A$ зависит от переменных $p_1, \ldots, p_n$. Тогда по лемме $p_1, \ldots, p_n \vdash A$. И $p_1, \ldots, \lnot p_n \vdash A$. И вообще как угодно можно расставить отрицания, потому что $A$ --- тавтология. Из двух приведённых фактов по производному правилу $p_1, \ldots, p_{n - 1}, p_n \lor \lnot p_n \vdash A$. Но $p_n \lor \lnot p_n$ можно получить из аксиомы, значит это можно выкинуть из списка гипотез и получить $p_1, \ldots, p_{n - 1} \vdash A$. Аналогично начав с $p_1, \ldots, \lnot p_{n - 1}, p_n \vdash A$ и $p_1, \ldots, \lnot p_{n - 1}, \lnot p_n \vdash A$, мы получим $p_1, \ldots, \lnot p_{n - 1} \vdash A$. Из этих двух результатов мы сможем избавиться от $p_{n - 1}$ и получить $p_1, \ldots, p_{n - 2} \vdash A$. Долго повторяя этот процесс, мы избавимся от всех переменных и получим $\ \vdash A$, а это то, что требовалось.  
\end{proof}

\section{Исчисление резолюций для опровержения пропозициональных формул в конъюнктивной нормальной форме (КНФ): дизъюнкты, правило резолюции, опровержение КНФ в исчислении резолюций. Теорема корректности исчисления резолюций (для пропозициональных формул в КНФ)}
Если исчисление высказываний работало с произвольными формулами, построенными с помощью отрицания, конъюнкции, дизъюнкции и импликации, исчисление резолюций работает только с дизъюнктами.

\begin{definition} \textit{Литерал} --- переменная или отрицание переменной \end{definition}
\begin{definition} \textit{Дизъюнкт} --- это дизъюнкция по некоторому конечному множеству литералов \end{definition}

Обратите внимание, что в этом определении речь про множество. Хотя мы записываем дизъюнкты как формулы $\lambda_1 \lor \lambda_2 \lor \ldots \lor \lambda_n$, мы считаем, что, к примеру, $\lambda_1 \lor \lambda_2, \lambda_2 \lor \lambda_1, \lambda_1 \lor \lambda_2 \lor \lambda_1$ --- это всё один и тот же дизъюнкт.

У исчисления резолюций нет аксиом и есть одно правило --- правило резолюции
\[\tfrac{A \lor p, \; B \lor \lnot p}{A \lor B} \]
Отметим, что при применении правила к $p$ и $\lnot p$ результатом будет пустой дизъюнкт, который обозначается как $\perp$(или как $\square$). 

На записанные в КНФ пропозиональные формулы можно смотреть как на множества дизъюнктов в исчислении резолюций. Будем говорить, что множество дизъюнктов совместно, если есть набор значений переменных, при котором каждый дизъюнкт возвращает истину. Утверждается, что из множества дизъюнктов можно вывести в исчислени пустой дизъюнкт, если и только если множество несовместно.

\begin{theorem}{(Корректность исчисления резолюций)}
	Если множества дизъюнктов можно вывести пустой дизъюнкт, то оно несовместно.
\end{theorem}
\begin{proof}
	Можно убедиться, что из истинных (при каких-то значениях переменных) формул можно вывести только истинные (при тех же значениях). Но пустой дизъюнкт всегда ложен.

	Если вам по каким-либо причинам не нравятся слова о ложности пустого дизъюнкта, можно сказать, что пустой дизъюнкт можно вывести только из $p, \lnot p$, а они не могут быть истинны одновременно.
\end{proof}

\section{Теорема полноты исчисления резолюций (для пропозициональных формул в КНФ). Доказательство только для конечных и счетных множеств формул.}
\begin{theorem}{(Полнота исчисления резолюций)}
	Если множество дизъюнктов $S$ несовместно, то из него можно вывести пустой дизъюнкт.
\end{theorem}
Докажем для случая, когда $S$ не более чем счётно.
\begin{proof}
	Применим контрапозицию и докажем, что если из $S$ нельзя вывести пустой дизъюнкт, то оно совместно. Обозначим за $S'$ множество всех формул, которые можно вывести из $S$. Поскольку множество дизъюнктов не более чем счётно, а сами дизъюнкты конечны, множество используемых переменных тоже будет не более чем счётно. Занумеруем их $x_1, x_2, \ldots$. Докажем, что можно так выбрать значения переменным, что для любого $n$ все дизъюнкты из $S'$, содержащие только переменные с номерами не больше $n$, истинны. Ясно, что это и означает совместность. Доказывать будем индукцией по $n$.
	\paragraph{База индукции.} Это могло бы быть неверно для $n = 1$, только если бы в $S'$ содержались $x_1$ и $\lnot x_1$. Но такого быть не может, ведь тогда мы могли бы вывести пустой дизъюнкт.
	\paragraph{Шаг индукции.} По предположению индукции мы уже как-то умеем выбирать значения для переменных $x_1, \ldots, x_n$. Предположим, выбрать значение для $x_{n + 1}$ нельзя.
	\begin{itemize}
	\item $x_{n + 1} = 0$ не подходит $\implies A \lor x_{n + 1} \in S'$, где $A$ содержит только $x_1,\ldots, x_n$ и ложно при выбранных для них значениях.
	\item $x_{n + 1} = 1$ не подходит $\implies B \lor \lnot x_{n + 1} \in S'$, где $B$ содержит только $x_1,\ldots, x_n$ и ложно при выбранных для них значениях.
	\end{itemize}
	Но тогда можно вывести $A \lor B$. Поскольку $A \lor B$ содержит только $x_1, \ldots, x_n$, по предположению индукции оно верно при выбранных значениях. А значит не может быть, что и $A$, и $B$ ложны, противоречие. 
\end{proof}
На самом деле аналогичным образом можно было бы доказать корректность и для несчётных множеств формул (для корректности индукции пришлось бы прибегнуть к теореме Цермело), но несчётное число переменных --- это крайне нетипичная ситуация и этим мы тут не занимаемся.

\section{Полиномиальный алгоритм сведения задачи распознавания совместности конечных множеств произвольных формул к задаче распознавания совместности конечных множеств дизъюнктов.}
Перейти от формулы к конечному множеству дизъюнктов --- то же самое, что привести её к КНФ. Ясно, что формулу длины $l$, зависящую от $m$ переменных, можно за $O(2^m \cdot l)$ --- построить таблицу истинности и взять дизъюнкты, соответствующие строкам, в которых формула ложна, но этот метод не полиномиальный.

Пусть наша формула имеет вид $f(A, B)$ --- где $A, B$ --- некоторые формулы, а $f$ --- операция, выполняющаяся в нашей формуле последней. Тогда введём новые переменные $x', x''$ и заменим нашу формулу на $f(x', x'') \land (x' \equiv A) \land (x'' \equiv B)$. Длины формулы $f(x', x'')$ ---константа, следовательно экспоненциальный метод сведёт её к КНФ за $O(1)$. Повторим процедуру для формул $x' \equiv A$ и $x'' \equiv B$, если $A$ и $B$ -- это не просто переменные. За каждый запуск наивного алгоритма мы избавляемся от одной операции в исходной формуле, поэтому время работы можно оценить как $O(l)$.

Мы дали описание для бинарных операций, но ясно, что тот же самый подход замены выражений на переменные применим и для отрицания, и для каких-то экзотических операций.
\end{document}
