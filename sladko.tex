\documentclass{article}
\usepackage{packages}

\newcommand{\im}{\mathrm{Im}}

\title{Коллоквиум по Дискретной математике, 2 курс}
\author{Залялов Александр, @bcategorytheory,\\  Солодовников Никита, @applied\_memes \\ Шморгунов Александр, @Owlus }
\date{}

\begin{document}

\maketitle

\setcounter{section}{9}
\section{Определение машин Тьюринга и вычислимых на машинах Тьюринга функций. Тезис Чёрча-Тьюринга. Неразрешимость проблемы остановки машины Тьюринга.}
Машина Тьюринга задаётся\footnote{Здесь машина Тьюринга определяется в соответсвии с лекцией. Следует понимать, что это определение не является общепринятым. Вариаций масса: кто-то запрещает головке оставаться на месте, кто-то выделяет выходной алфавит, отличный от входного и т. д.}
\begin{itemize}
	\item непустым конечным алфавитом $\Sigma$, среди которого выделен пробельный символ $\vartextvisiblespace$ и не содержащее пробела подмножество $\Gamma$ --- входной алфавит;
	\item непустым конечным множеством состояний $Q$, среди которых выделено начальное состояние $s_0$ и множество терминальных состояний $F$;
	\item функцией переходов $\delta:(Q \setminus F) \times \Sigma \to Q \times \Sigma \times \{-1, 0, +1\}$.
\end{itemize}

Машина Тьюринга состоит из бесконечной ленты, разбитой на ячейки, головки, в любой момент времени указывающей на одну ячейку и одной ячейки памяти, в которой хранится текущее состояние. В начальный момент времени на ленте записано некоторое слово, составленное из букв входного алфавита, головка смотрит на первый символ этого слова, во всех остальных ячейках пробелы. Затем в каждый момент времени вычисляется $\delta(q, c) = (q', c', \Delta)$, где $q$ --- текущее состояние, $c$ --- символ записанный в ячейке, на которую сейчас смотрит головка. Состояние меняется на $q'$, символ в текущей ячейке на $c'$, головка остаётся на месте или передвигается на один влево или вправо в соответствии со значением $\Delta$. Если $q'$ оказалось терминальным, на этом работа машины заканчивается, иначе этот процесс продолжается.

Машины Тьюринга естественным образом отождествляются с частичными функциями $f:\Gamma^* \to \Gamma^*$ --- аргументом функции является входное слово, а возвращает функция слово, записанное на ленте после завершения работы машины(то есть всё, что написано на ленте, кроме бесконечного числа пробелов слева и справа). Функции будут частичными, поскольку машина Тьюринга может продолжать работать бесконечно или в данной конструкции может оказаться, что на выходе есть символ, не содержащийся в $\Gamma$. Функции, которые можно таким образом получить по некоторой машине Тьюринга, называются вычислимыми на машине Тьюринга.
\paragraph{Тезис Чёрча-Тьюринга.} \textit{Любая вычислимая функция вычислима на машине Тьюринга.} 

Здесь понятие "вычислимая функция" используется в неформальном смысле, под ним понимается функция, вычислимая в любой разумной модели, которая может прийти вам в голову. Тезис не является формальным утверждением, он никак не доказывается и принимается нами на веру.

\begin{theorem}
	Не существует вычислимой функции, определяющей по машине Тьюринга и входному слову, остановится ли эта машина. 
\end{theorem}
Теперь, когда мы отождествили вычислимые и вычислимые на машине Тьюринга функции, эта теорема непосредственно следует из доказательства теоремы о существовании полного перечислимого множества из 7 билета.

\section{Неразрешимость проблемы достижимости в односторонних ассоциативных исчислениях. Полугруппы, заданные порождающими и соотношениями. Теорема Маркова–Поста: неразрешимость проблемы равенства слов в некоторой конечно определенной полугруппе (без доказательства).}

\begin{definition}
	\textit{Односторонним ассоциативным исчислением} называется множество из всех слов над некоторым конечным алфавитом и конечный набор подстановок. Каждая подстановка представляет собой пару слов $(s, t)$ и позволяет в любом слове содержащем $s$ как подстроку заменить её на $t$ (но не наоборот).
\end{definition}

\begin{theorem}
	Существует одностороннее ассоциатвиное исчисление, в котором не разрешима задача проверить по паре слов, можно ли некоторой последовательностью подстановок перейти от первого ко второму.
\end{theorem}

\begin{proof}
	Возьмём некоторую машину Тьюринга $M$, для которой неразрешима проблема остановки, при чём если такую, что если она останавливается, то на ленте записано пустое слово. Построим по ней одностороннее ассоциативное исчисление, в котором из $[X]$ можно получить $Y$, если и только если $M$ преобразует $X$ в $Y$. В качестве алфавита для исчисления возьмём объединение алфавита $M$ и её множества состояний (а также квадратные скобки и символы $\triangleleft, \triangleright$). Будем сопоставлять конфигурациям машины слова исчисления. Если машина находится в состоянии $s$, на ленте записано слово $PQ$(конкатенация слов $P$ и $Q$) и головка указывает на первый символ слова $Q$, сопоставим такой конфигурации слово $[PsQ]$ в нашем исчислении. Тут важно, что мы считаем, что у машины не пересекаются алфавит и множество состояний. Построим по переходам машины подстановки для исчисления.
	
	\begin{center} \begin{tabular}{c | c }
		Переход МТ & Подстановка одностороннего ассоциативного исчисления \\ \hline
		$(s, c) \mapsto (s', c', 0)$ & $sc \to s'c'$ \\
		$(s, c) \mapsto (s', c', +1)$ & $sc \to c's'$ \\
		$(s, c) \mapsto (s', c', -1)$ & $xsc \to s'xc'$ --- для каждого символа $x$ из алфавита машины, а также $[sc \to [s'\vartextvisiblespace c'$ \\
		$(s, \vartextvisiblespace) \mapsto (s', c', 0)$ & $s] \to s'c']$ \\
		$(s, \vartextvisiblespace) \mapsto (s', c', +1)$ & $s] \to c's']$ \\
		$(s, \vartextvisiblespace) \mapsto (s', c', -1)$ & $xs] \to s'xc']$
	\end{tabular} \end{center}

	Дополнительно к этому введём подстановки, позволяющие получить пустое слово, если машина остановится.
	\begin{itemize}
		\item $f \to \triangleleft$, $f$ --- терминальное состояние;
		\item $c\triangleleft \to \triangleleft, c \ne [$;
		\item $[\triangleleft \to \triangleright$;
		\item $\triangleright c \to \triangleright, c \ne ]$;
		\item $\triangleright ] \to \varepsilon$(пустое слово).
	\end{itemize}

	Это можно было бы реализовать проще без двух дополнительных символов, но так мы получаем, что всегда существует ровно одна последовательностей подстановок, моделирующих работу машины Тьюринга. Осталась одна деталь --- мы пообщеали, что мы начнём с $[X]$, а не с $[s_0X]$. Она решается просто --- добавлением подстановки $[x \to [s_0x$ для всех символом $x$ из алфавита машины. 

	Итак, мы свели задачу остановки машины Тьюринга (про которую было известно, что она неразрешима) к задаче достижимости в одностороннем ассоциативном исчислении и показали этим, что эта задача тоже неразрешима.
\end{proof}
Оказывается, если потребовать, чтобы все подстановки были двухсторонними, то задача останется неразрешимой, но доказывать этот факт от нас не требуют. При чём такую задачу можно сформулировать на языке алгебры:

Пусть про некоторую полугруппу известно, что она содержит элементы $a_1, \ldots, a_n$ и в ней выполняются некоторые (конечное количество) равенства вида $a_{i_1}a_{i_2}\ldots a_{i_k} = a_{j_1}a_{j_2}\ldots a_{j_m}$. Обязательно ли в ней выполняется заданное равенство такого же вида?

\end{document}
