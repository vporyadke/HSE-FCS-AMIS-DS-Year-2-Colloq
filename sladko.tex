\documentclass{article}
\usepackage{packages}

\newcommand{\im}{\mathrm{Im}}

\title{Коллоквиум по Дискретной математике, 2 курс}
\author{Залялов Александр, @bcategorytheory,\\  Солодовников Никита, @applied\_memes \\ Шморгунов Александр, @Owlus }
\date{}

\begin{document}

\maketitle

\setcounter{section}{19}
\section{Дизъюнкты, универсальные дизъюнкты. Исчисление резолюций (ИР) для доказательства несовместности множеств универсальных дизъюнктов. Теорема корректности ИР.}
\begin{definition}
	\textit{Дизъюнктом} называется дизъюнкция атомарных формул и их отрицаний.\\ 
	Пример: $P(x) \vee \bar Q(y, f(x)) \vee s$ (в данном случае $P(x), \bar Q(y, f(x)), s$ --- атомарные формулы, а $\vee$ --- операция дизъюнкции)
\end{definition}
\begin{definition}
	\textit{Универсальным дизъюнктом} называется формула, полученная из дизъюнкта приписыванием кванторов всеобщности. Пример: $\forall x\forall y [P(x) \vee \bar Q(y, f(x)) \vee s]$
\end{definition}
Для того, чтобы доказывать несовместность множеств универсальных дизъюнктов (несовместность конъюнкции всех универсальных дизъюнктов этого множества) мы пользуемся правилами в исчислении резолюций
\vspace{2mm}

Правил в исчислении резолюций два:
\begin{itemize}
	\item $A \vee p,\ B \vee \bar p \rightarrow A \vee B$ (правило резолюций)
	\item $\forall x D(x) \rightarrow D(t)$ для некоторого $t$ 
\end{itemize}

\begin{theorem}{(Теорема корректности исчисления резолюций)}
	Если из набора универсальных дизъюнктов можно вывести пустой дизъюнкт, то этот набор несовместен.
\end{theorem}
\begin{proof}
	Идем методом от противного. Пускай существует модель $M$, в которой все данные дизъюнкы истинны.
	Заметим, что оба правила исчисления резолюций сохраняют истинность.
	
	Действительно, если в правиле резолюций для $A \vee p,\ B \vee \bar p \rightarrow A \vee B$ мы получим, что $A \vee B$ ложно, то это значит, что и $A$, и $B$ ложно, однако если это так, то ложно либо $A \vee p$, либо $B \vee \bar p$
	
	В правиле резолюций для $\forall x D(x) \rightarrow D(t)$, если выражение $D(x)$ истинно для любого $x$, то оно будет истинно для и для некоторого $t$ \footnote{В случае $\forall x \exist y, x < y$ не выполняется, если мы поставим $x = y$. Однако наши универсальные дизъюнкты не допускают квантора существования, поэтому такая формула невозможна}
	
	Если мы вывели пустой дизъюнкт, то по истинности правил исчисления резолюций получаем, что пустой дизъюнкт является истинным. Противоречие.
\end{proof}

\section{Непротиворечивые теории. Теорема полноты ИР (для множеств универсальных дизъюнктов).}

\begin{definition}
	\textit{Непротиворечивой теорией} называется теория такая, что в ней утверждение не может быть одновременно доказано и опровергнуто
\end{definition}

\begin{theorem}{(Теорема полноты исчисления резолюций)}
	Если набор универсальных дизъюнктов несовместен, то из него можно вывести пустой дизъюнкт
\end{theorem}
\begin{proof}
	Пускай есть счётное множество универсальных дизъюнктов $S$. Заметим, что если мы подставим вместо терм конкретные значения в этом множестве, то можем заменить все атомарные формулы пропозициональными переменными (константа 0 исчезает, так как дизъюнкция; константу 1 заменяем на дизъюнкцию переменных $p \vee \bar p$, $p$ в данном случае --- новая переменная, которую мы ввели). Назовём это новое множество из пропозициональных переменных $S'$.
	
	Так как $S$ несовместно, то $S'$ тоже несовместно, так как существует набор терм, при котором из $S$ мы получаем $S'$. Значит, мы свели теорему к случаю для дизъюнктов из пропозициональных переменных. Теорема полноты для такого случая доказывалась в билете 16. А раз для любого набора терм вывести пустой дизъюнкт нельзя, то и для $S$ тоже. 
\end{proof}
\end{document}