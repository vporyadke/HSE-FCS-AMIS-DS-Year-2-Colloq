\documentclass{article}
\usepackage{packages}

\title{Коллоквиум по Дискретной математике, 2 курс}
\author{Залялов Александр, @bcategorytheory,\\  Солодовников Никита, @applied\_memes \\ Шморгунов Александр, @Owlus }
\date{}

\begin{document}

\maketitle

    \section{Определение вычислимой частичной функции из $\mathbb{N}$ в $\mathbb{N}$.
    Счётность семейства частичных вычислимых функций, и существование невычислимых функций.
    Разрешимые и перечислимые подмножества $\mathbb{N}$. Счётность семейства перечислимых множеств,
    существование неперечислимых множеств.}

    Здесь не даётся формального определения алгоритма. В нашем случае ``алгоритм'' --- некоторый
    чёрный ящик, принимающий на вход конструктивный объект (натуральное число или же объект, который
    можно закодировать как натуральное число), производящий некоторый результат (также конструктивный
    объект), а также работающий по шагам (некоторые атомарные действия вроде сложения).

    Rule of thumb для вопроса ``Алгоритм ли это?'' --- можно написать как программу на каком-нибудь
    языке программирования.

    \begin{definition}
        Функция $f: \mathbb{N} \to \mathbb{N}$ называется \textit{частичной}, если $\operatorname{Dom} f \subseteq \mathbb{N}$.
    \end{definition}

    \begin{definition}
        Функция $f: \mathbb{N} \to \mathbb{N}$ называется \textit{тотальной}, если $\operatorname{Dom} f = \mathbb{N}.$
    \end{definition}

    \begin{definition}
        Алгоритм $\mathcal{A}$ \textit{вычисляет} частичную функцию $f: \mathbb{N} \to \mathbb{N}$, если
        \begin{gather*}
            \begin{cases}
                \mathcal{A}(x) = f(x), & \text{если $x \in \operatorname{Dom} f$,}\\
                \mathcal{A}(x) \text{ не определено} & \text{иначе.}
            \end{cases}
        \end{gather*}
    \end{definition}

    \begin{definition}
        Частичная функция $f: \mathbb{N} \to \mathbb{N}$ называется \textit{вычислимой}, если существует
        алгоритм, её вычисляющий.
    \end{definition}

    \begin{statement}
        Множество частичных вычислимых функций не более, чем счётно.
    \end{statement}
    \begin{proof}
        Действительно, всякой вычислимой функции можно поставить в соответствие некоторый алгоритм,
        причём различные функции вычисляются различными алгоритмами. Алгоритм --- это программа,
        то есть конечная строка. Конечных строк (а следовательно, алгоритмов) всего лишь счётное число.
        Существует инъекция из множества вычислимых функций в множество алгоритмов, значит, количество
        вычислимых функций не более, чем счётно.
    \end{proof}

    \begin{theorem}
        Существуют невычислимые функции $f: \mathbb{N} \to \mathbb{N}$.
    \end{theorem}
    \begin{proof}
        Мощность множества вычислимых функций меньше мощности множества всех функций из $\mathbb{N}
        \to \mathbb{N}$, а значит, его дополнение не пусто.
    \end{proof}

    \begin{definition}
        Множество $A \subseteq \mathbb{N}$ называется \textit{разрешимым}, если существует алгоритм, вычисляющий
        его характеристическую функцию $\chi_A(x)$, то есть функцию такую, что
        \begin{gather*}
            \chi_A(x) =
            \begin{cases}
                1, & \text{если $x \in A$,}\\
                0 & \text{иначе.}
            \end{cases}
        \end{gather*}
    \end{definition}

    \begin{definition}
        Множество $A \subseteq \mathbb{N}$ называется \textit{перечислимым}, если существует
        алгоритм (не принимающий никаких входных данных), который выводит последовательность $a_n$
        такую, что множество всех элементов этой последовательности равно $A$.
    \end{definition}

    \begin{statement}
        Множество перечислимых подмножеств $\mathbb{N}$ не более, чем счётно.
    \end{statement}
    \begin{proof}
        Всякому перечислимому множеству соответствует алгоритм, его перечисляющий, причём различные множества
        перечисляются различными алгоритмами. Отсюда следует, что мощность множества перечислимых множеств не
        превосходит мощности множества алгоритмов, которое, в свою очередь, является счётным.
    \end{proof}

    \begin{theorem}
        Существуют неперечислимые множества $A \subseteq \mathbb{N}$.
    \end{theorem}
    \begin{proof}
        Множество перечислимых множеств имеет мощность меньшую, чем $2^\mathbb{N}$. Значит, его дополнение
        не пусто.
    \end{proof}

    \section{Эквивалентные определения перечислимости: полуразрешимость, область определения
    вычислимой функции, множество значений вычислимой функции.}

    \begin{definition}
        Множество $A \subseteq \mathbb{N}$ называется \textit{полуразрешимым}, если существует алгоритм,
        вычисляющий его полухарактеристическую функцию $\xi_A(x)$, то есть функцию такую, что
        \begin{gather*}
            \xi_A(x) =
            \begin{cases}
                1, & \text{если $x \in A$,}\\
                \text{не определено} & \text{иначе.}
            \end{cases}
        \end{gather*}
    \end{definition}

    \begin{theorem}
        Следующие утверждения эквивалентны:
        \begin{enumerate}
            \item Множество $A$ перечислимо.
            \item Множество $A$ полуразрешимо.
            \item Существует частичная вычислимая функция $f: \mathbb{N} \to \mathbb{N}$ такая, что
                $A = \operatorname{Dom} f$.
            \item Существует частичная вычислимая функция $f: \mathbb{N} \to \mathbb{N}$ такая, что
                $A = \operatorname{Ran} f$.
        \end{enumerate}
    \end{theorem}

    \begin{proof}
        Чтобы показать эквивалентность всех этих утверждений, докажем несколько импликаций.

        $(1) \implies (2)$

        Множество $A$ перечислимо, докажем его полуразрешимость.

        Модифицируем алгоритм $\mathcal{A}$ перечисления множества $A$ следующим образом: если для входа $x$
        при перечислении мы встретили $x$, вернём ответ $1$, иначе продолжим работу, ничего не возвращая.
        Так как в последовательности, получаемой алгоритмом, рано или поздно встретится каждый из
        элементов $A$, положительный ответ будет дан за конечное число шагов. В случае, если $x
        \not\in A$, алгоритм зациклится без вывода, что вполне устраивает нас в рамках нашей задачи.

        $(2) \implies (3)$

        Множество $A$ полуразрешимо, докажем, что найдётся вычислимая функция, для которой $A$ ---
        область значений.

        Этой частичной вычислимой функцией был \sout{Альберт Эйнштейн} $\xi_A(x)$. Действительно,
        знаем, что $\xi_A(x)$ вычислима, а $\operatorname{Dom} \xi_A = A$. Значит, мы
        нашли искомую функцию.

        $(3) \implies (4)$

        Множество $A$ является областью определения некоторой вычислимой функции $f$, докажем, что
        оно также является областью значений некоторой другой вычислимой функции.

        Определим функцию $g(x)$:
        \begin{gather*}
            g(x) =
            \begin{cases}
                x, & \text{если } x \in \operatorname{Dom} f,\\
                \text{не определено} & \text{иначе.}
            \end{cases}
        \end{gather*}

        Эта функция вычислима. Алгоритм, её вычисляющий, должен попытаться вычислить $f(x)$ и затем
        просто вывести $x$. Кроме того, $\operatorname{Ran} g = \operatorname{Dom} f$, а значит, мы
        нашли искомую функцию.

        $(4) \implies (1)$

        Множество $A$ является областью значений некоторой вычислимой функции, докажем его
        перечислимость.

        Известно, что существует алгоритм $\mathcal{F}$, вычисляющий функцию $f$, область значений
        которой совпадает с $A$. Чтобы перечислить элементы множества $A$, будем бесконечно
        производить итерации следующего вида: на $n$-той итерации запустим по очереди на $n$
        шагов $\mathcal{F}(i)$ для каждого $0 \leqslant i \leqslant n$. Таким образом, для всех $i$
        алгоритм $\mathcal{F}(i)$ будет рано или поздно запущен на число шагов, необходимое для
        завершения.  Значит, на всех $x$ из $\operatorname{Dom} f$ мы вычислим (и выведем) $f(x)$.
        Таким образом будут выведены все элементы $\operatorname{Ran} f$, равного $A$.

        Из каждого утверждения следуют все остальные. Значит, утверждения эквивалентны.
    \end{proof}

    \section{Теорема Поста. Теорема о графике.}

    \paragraph{Теорема Поста.} Если множества $A$ и $\mathbb{N} \setminus A$ перечислимы, то $A$
    разрешимо.

    \begin{proof}
        Перечислимость множества эквивалентна его полуразрешимости. Будем использовать алгоритмы
        $\mathcal{A}$ и $\mathcal{B}$, вычисляющие $\xi_{A}(x)$ и $\xi_{\mathbb{N} \setminus A}(x)$
        соответственно.

        Алгоритм $\mathcal{C}$, находящий $\chi_A(x)$, будет по очереди запускать $\mathcal{A}(x)$ и
        $\mathcal{B}(x)$ на некоторое число шагов. Как только один из этих алгоритмов завершится, можно
        будет дать ответ: если $x \in A$, вернуть $1$, в противном случае вернуть $0$.

        Докажем корректность построенного алгоритма. Во-первых, он действительно даёт правильный
        ответ на всех $x \in \mathbb{N}$, а во-вторых, всегда завершается, так как всякое
        натуральное число лежит либо в множестве $A$, либо в его дополнении. Оба вспомогательных
        алгоритма могут при необходимости отработать бесконечное число шагов, значит, если какой-то
        из них завершается на данном входе, он завершится.
    \end{proof}

    \begin{definition}
        Пусть задана функция $f$. Множество $\Gamma_f = \{(x, f(x)) \mid x \in \operatorname{Dom} f\}$
        называется \textit{графиком} функции $f$.
    \end{definition}

    \paragraph{Теорема о графике.} Функция $f$ вычислима тогда и только тогда, когда $\Gamma_f$
    перечислимо.
    \begin{proof}\

        $\implies$

        Умея вычислять функцию $f$, хотим перечислить $\Gamma_f$.

        Будем бесконечно производить итерации следующего вида: на $n$-той итерации попытаемся
        вычислить $f(x)$ для всех $0 \leqslant x \leqslant n$ не более, чем за $n$ шагов.
        Если удастся, выведем пару $(x, f(x))$ в противном случае остановим вычисление $f(x)$
        и перейдём к следующему $i$. Для каждого $x \in \operatorname{Dom} f$ рано или поздно мы
        произведём достаточное число шагов, чтобы вычислить $f(x)$, так как алгоритм, вычисляющий
        $f(x)$, должен завершаться за конечное число шагов. Значит, $\Gamma_f$ таким образом
        действительно будет перечислено.

        $\impliedby$

        Умея перечислять $\Gamma_f$, хотим вычислить $f(x)$.

        Будем перечислять $\Gamma_f$, пока не найдём пару, в которой первый элемент равен $x$.
        Действительно, если функция определена на $x$, то такая пара найдётся в $\Gamma_f$,
        а значит, будет выведена алгоритмом его перечисления за конечное число шагов. Далее просто
        выведем второй элемент этой пары и завершим работу.

    \end{proof}

    \section{Универсальные вычислимые функции (нумерации) для семейства частичных вычислимых
    функций натурального аргумента. Несуществование универсальной вычислимой функции для
    семейства тотальных вычислимых функций натурального аргумента (диагональное рассуждение).
    Главные универсальные функции.}

    \section{Вычислимая функция, не имеющая тотального вычислимого продолжения. Перечислимое неразрешимое
    множество. Неразрешимость проблемы применимости.}

    \section{Теорема Поста. Существование перечислимого множества, дополнение которого неперечислимо.
    Перечислимые неотделимые множества.}

    \section{Сводимости: m-сводимость и Тьюрингова сводимость. Свойства. Полные перечислимые множества.}

    \section{Теорема Клини о неподвижной точке.}

    \section{Теорема Райса-Успенского.}

\end{document}
