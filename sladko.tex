\documentclass{article}
\usepackage{packages}

\title{Коллоквиум по Дискретной математике, 2 курс}
\author{Залялов Александр, @bcategorytheory,\\  Солодовников Никита, @applied\_memes, \\ Шморгунов Александр, @Owlus \\
Виноградова Дарья, @orange\_to\_the\_wall}
\date{}

\begin{document}

\maketitle

\setcounter{section}{25}
\section{Игры Эренфойхта} 

Цель: сформулировать общий критерий
элементарной эквивалентности двух интерпретаций некоторой сигнатуры (считаем, что сигнатура содержит только предикатные символы).



Критерий будет сформулирован в терминах некоторой игры, называемой игрой Эренфойхта. В ней участвуют два игрока, называемые Новатором (Н) и Консерватором (К). Игра определяется выбранной парой интерпретаций.

В начале игры Новатор объявляет натуральное число $k$. Далее
они ходят по очереди, начиная с Н; каждый из игроков делает $k$ ходов, после чего определяется победитель.

На $i$-м ходу Н выбирает элемент в одной из интерпретаций (в
любой из двух) и помечает его числом i. В ответ К выбирает некоторый элемент из другой интерпретации и также помечает его числом i.

После k ходов игра заканчивается. При этом в каждой интерпретации k элементов оказываются помеченными числами от $1$ до $k$ (мы не учитываем, кто именно из игроков их пометил). Обозначим эти элементы
$a_1$, $a_2$, $\cdots$ , $a_k$ (для первой интерпретации)
и $b_1$, $b_2$, $\cdots$ , $b_k$ (для второй). Элементы $a_i$ и $b_i$ (с одним и тем же i) будем называть соответствующими друг другу.

Посмотрим, найдётся ли предикат сигнатуры, который различает помеченные элементы первой и второй интерпретации (то есть истинен на некотором наборе помеченных элементов в одной интерпретации, но ложен на соответствующих элементах другой). Если такой предикат найдётся, то выигрывает Новатор, в противном случае — Консерватор.

\begin{theorem}
Интерпретации не элементарно эквивалентны $\Longleftrightarrow$ Н имеет выигрышную стратегию в этой игре.
\end{theorem}

\begin{proof}

Докажем, что если Новатор имеет выигрышную стратегию, то интерпретации не элементарно эквивалентны.

Пусть есть различающая формула. Приведем ее к предваренной форме. Будем последовательно смотреть на кванторы в ее начале. Пусть текущий квантор - это $\exists$. Значит, есть элемент в $M_1$, для которого верна оставшаяся часть формулы, в то время как в $M_2$ такого нет. Этот элемент и должен выбрать Новатор очередным ходом.

Пусть текущий квантор - это $\forall$. В таком случае мы можем перейти к отрицанию и поступить аналогично шагу с $\exists$, только выбирая элемент в $M_2$.

Таким образом, за количество шагов, равное количеству кванторов в различающей формуле, Новатор может построить различающие наборы.

\end{proof}

\setcounter{section}{26}
\section{Семантически полные теории. Критерий семантической полноты теории в терминах эле-
ментарной эквивалентности моделей. Аксиоматизация элементарной теории упорядоченного
множества целых чисел.}

\begin{definition}\textit{Аксиоматическая теория T} - множество замкнутых формул.
\end{definition}

T \textit{семантически полна}, если для любой замкнутой формулы А выполнено одно из двух:
\begin{enumerate}
    \item из T семантически следует А (А истинно во всех моделях теории)
    \item из Т семантически следует $\neg A$
\end{enumerate}

\begin{lemma}
Теория семантически полна $\Longleftrightarrow$ любые 2 ее модели элементарно эквивалентны.
\end{lemma}
\begin{proof}

$\Rightarrow$ Элементарная эквивалентность значит, что в обоих моделях любая формула или истинна, или ложна. Тогда если $\phi$ следует из А, то она истинна для всех моделей, следовательно, для каждой пары. Аналогично для $\neg \phi$

$\Leftarrow$ От противного: какая-то формула сама не следует и ее отрицание не следует. Значит, есть модели, в одной из которых А истинно, в другой - ложно. Противоречие с элементарной эквивалентностью.

\end{proof}

\noindent\textbf{Аксиоматизация множества рациональных чисел}
\vspace{2mm}

$M = ( \mathbb{Q}, =, <)$
\begin{itemize}
    \item аксиомы равенства
    \begin{enumerate}
        \item $\forall x \: x = x$
        \item $\forall x \forall y \: x = y \rightarrow y = x$
        \item $\forall x \forall y \forall z \: \: x = y \land y = z \rightarrow x = z$
        \item $\forall x_1 \forall x_2 \forall y_1 \forall y_2 \: \: x_1 = x_2 \land y_1 = y_2 \rightarrow (x_1 = x_2 \rightarrow y_1 = y_2)$
    \end{enumerate}
    \item аксиомы линейного порядка
    \begin{enumerate}
        \item $x < y \land y < z \rightarrow x < z$
        \item $\neg \: (x < x)$
        \item $\forall x \forall y \: x < y \lor x > y \lor x = y$
    \end{enumerate}
    \item отсутствие наибольшего и наименьшего элемента
    \item плотность множества $\forall x, y \: (x < y \rightarrow \exists z \: \:  x < z \land z < y)$
\end{itemize}

\begin{theorem}{T - совместная и семантически полная.}
\begin{proof}
Доказательство аналогично игре Эренфойхта с  \mathbb{R} и  \mathbb{Q}. Все выбранные в одной модели элементы идут в том же порядке, что и элементы второй модели. Консерватору достаточно возможности выбрать элемент между любыми двумя и отсутствие наибольшего и наименьшего элемента. 
\end{proof}
\end{theorem}

\setcounter{section}{27}
\section{Аксиоматизация множества целых чисел.}

$M = (\mathbb{Z}, =, <)$
\begin{itemize}
    \item аксиомы равенства
    \item аксиомы линейного порядка
    \item отсутствие наибольшего и наименьшего элемента
    \item $\forall x \exists y (x < y \land \neg (\exists z \: \: x < z \land z < y))$
    \item $\forall x \exists y (x > y \land \neg (\exists z \: \: x > z \land z > y))$
\end{itemize}

\theorem{T - совместная и семантически полная.}
\begin{proof}
Как устроены модели Т? Это  \mathbb{Z},  \mathbb{Z}+\mathbb{Z} или любое множество вида AZ (А - линейно упорядоченное множество, в каждом элементе которого лежит множество целых чисел). 
Скажем, что элементы эквивалентны, если мы можем получить один из другого за конечное число шагов. Факторизуем по этому отношению эквивалентности.
\end{proof}

\begin{lemma}Для любого линейно упорядоченного $A \: \: A  \mathbb{Z} \cong  \mathbb{Z}$
\begin{proof}
Доказывается аналогично случаю с  \mathbb{Z}+\mathbb{Z}
\end{proof}
\end{lemma}
\end{theorem}

\end{document}
