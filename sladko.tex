\documentclass{article}
\usepackage{packages}

\newcommand{\im}{\mathrm{Im}}

\title{Коллоквиум по Дискретной математике, 2 курс}
\author{Залялов Александр, @bcategorytheory,\\  Солодовников Никита, @applied\_memes \\ Шморгунов Александр, @Owlus }
\date{}

\begin{document}

\maketitle

\setcounter{section}{19}
\section{Дизъюнкты, универсальные дизъюнкты. Исчисление резолюций (ИР) для доказательства несовместности множеств универсальных дизъюнктов. Теорема корректности ИР.}
\begin{definition}
	\textit{Дизъюнктом} называется дизъюнкция атомарных формул и их отрицаний.\\ 
	Пример: $P(x) \vee \bar Q(y, f(x)) \vee s$ (в данном случае $P(x), \bar Q(y, f(x)), s$ --- атомарные формулы, а $\vee$ --- операция дизъюнкции)
\end{definition}
\begin{definition}
	\textit{Универсальным дизъюнктом} называется формула, полученная из дизъюнкта приписыванием кванторов всеобщности. Пример: $\forall x\forall y [P(x) \vee \bar Q(y, f(x)) \vee s]$
\end{definition}
Для того, чтобы доказывать несовместность множеств универсальных дизъюнктов (несовместность конъюнкции всех универсальных дизъюнктов этого множества) мы пользуемся правилами в исчислении резолюций
\vspace{2mm}

Правил в исчислении резолюций два:
\begin{itemize}
	\item $A \vee p,\ B \vee \bar p \rightarrow A \vee B$ (правило резолюций)
	\item $\forall x D(x) \rightarrow D(t)$ для некоторого $t$ 
\end{itemize}

\begin{theorem}{(Теорема корректности исчисления резолюций)}
	Если из набора универсальных дизъюнктов можно вывести пустой дизъюнкт, то этот набор несовместен.
\end{theorem}
\begin{proof}
	Идем методом от противного. Пускай существует модель $M$, в которой все данные дизъюнкы истинны.
	Заметим, что оба правила исчисления резолюций сохраняют истинность.
	
	Действительно, если в правиле резолюций для $A \vee p,\ B \vee \bar p \rightarrow A \vee B$ мы получим, что $A \vee B$ ложно, то это значит, что и $A$, и $B$ ложно, однако если это так, то ложно либо $A \vee p$, либо $B \vee \bar p$
	
	В правиле резолюций для $\forall x D(x) \rightarrow D(t)$, если выражение $D(x)$ истинно для любого $x$, то оно будет истинно для и для некоторого $t$ \footnote{В случае $\forall x \exist y, x < y$ не выполняется, если мы поставим $x = y$. Однако наши универсальные дизъюнкты не допускают квантора существования, поэтому такая формула невозможна}
	
	Если мы вывели пустой дизъюнкт, то по истинности правил исчисления резолюций получаем, что пустой дизъюнкт является истинным. Противоречие.
\end{proof}

\section{Непротиворечивые теории. Теорема полноты ИР (для множеств универсальных дизъюнктов).}

\begin{definition}
	\textit{Непротиворечивой теорией} называется теория такая, что в ней утверждение не может быть одновременно доказано и опровергнуто
\end{definition}

\begin{theorem}{(Теорема полноты исчисления резолюций)}
	Если набор универсальных дизъюнктов несовместен, то из него можно вывести пустой дизъюнкт
\end{theorem}
\begin{proof}
	Пускай есть счётное множество универсальных дизъюнктов $S$. Заметим, что если мы подставим вместо терм конкретные значения в этом множестве, то можем заменить все атомарные формулы пропозициональными переменными (константа 0 исчезает, так как дизъюнкция; константу 1 заменяем на дизъюнкцию переменных $p \vee \bar p$, $p$ в данном случае --- новая переменная, которую мы ввели). Назовём это новое множество из пропозициональных переменных $S'$.
	
	Так как $S$ несовместно, то $S'$ тоже несовместно, так как существует набор терм, при котором из $S$ мы получаем $S'$. Значит, мы свели теорему к случаю для дизъюнктов из пропозициональных переменных. Теорема полноты для такого случая доказывалась в билете 16. А раз для любого набора терм вывести пустой дизъюнкт нельзя, то и для $S$ тоже. 
\end{proof}

\section{Исчисление резолюций для теорий, состоящих из формул общего вида (приведение к предваренной нормальной форме и сколемизация). Доказательства общезначимости с помощью ИР. Выводимость формулы в теории с помощью ИР. Теорема компактности.}
Высказываниями мы называем утверждения, которые либо истинны, либо ложны. При этом если $A, B$ являются высказываниями, то $\lnot A, A \lor B, A \land B, A \to B$ --- тоже высказывания. Из такого определения никак не следует, что высказывания вообще существуют, так что в любом применении исчисления высказываний также описывают некоторые атомарные высказывания. Но нам для доказательства общих фактов это никак не потребуется. Исчисление высказываний задаётся аксиомами и правилами вывода. У нас имеется 11 аксиом:
\begin{enumerate}
	\item $(A \to (B \to C)) \to ((A \to B) \to (A \to C))$
	\item $A \to (B \to A)$
	\item $A \land B \to A$
	\item $A \land B \to B$
	\item $A \to (B \to A \land B)$
	\item $A \to A \lor B$
	\item $B \to A \lor B$
	\item $(A \to C) \to ((B \to C) \to ((A \lor B) \to C))$
	\item $\lnot A \to (A \to B)$
	\item $(A \to B) \to ((A \to \lnot B) \to \lnot A)$
	\item $A \lor \lnot A$
\end{enumerate}
и 1 правило вывода(modus ponens)
\[\tfrac{A \to B, \; A}{B} \]

Вывод в исчислении высказываний --- это последовательность из операций двух видов
\begin{itemize}
	\item Подстановка в некоторую аксиому любых высказываний вместо $A, B, C$.
	\item Применение правила вывода. Если уже выведены $A\to B$ и $A$, можно вывести $B$
\end{itemize}

\begin{theorem}{(Корректность исчисления резолюций)}
	Любая формула, которую можно вывести в исчислении высказываний, истинна(тавтологична).
\end{theorem}
Здесь мы называем формулу истинной(тавтологичной), если она как булева формула верна при всех значениях входящих в неё переменных. Отметим, что только в этом контексте мы понимаем $\lnot, \lor, \land$ как привычные логические операции. С точки зрения исчисления высказываний, это просто какие-то символы, всё что мы про них знаем --- это аксиомы и правило.
\begin{proof}
	Достаточно убедиться, что все действия, который мы можем производить в ходе вывода не позволяют получить ложное выражение. Во-первых, все аксиомы истинны при любых значениях входящих в них переменных. Во-вторых, если $A \to B$ истинно и $A$ истинно, то $B$ истинно.
\end{proof}

\section{Гомоморфизмы, эпиморфизмы (сюръективные гомоморфизмы), изоморфизмы. Теорема о сохранении истинности при эпиморфизме. Изоморфные модели. Элементарно эквивалентные модели, элементарная эквивалентность изоморфных моделей.}
Пусть $\Gamma$ --- некоторое множество высказываний(гипотез). Тогда говорят, что формула $A$ выводится из $\Gamma$, если её можно вывести,  разрешая пользоваться не только аксиомами и правилом вывода, но и высказываниями из $\Gamma$. Можно сказать, что у нас появилась третья операция: бесплатно получить формулу из $\Gamma$. Обозначение: $\Gamma \vdash A$. В таких терминах можно сказать, что формула, которую можно вывести в исчислении высказываний,  выводится из пустого множества гипотез, обозначение: $\ \vdash A$.

\begin{lemma}
	$\ \vdash A \to A$
\end{lemma}

\begin{proof}
	\begin{enumerate}
		\item $A \to (A \to A)$(2 аксиома)
		\item $\lnot A \to (A \to A)$ (9 аксиома)
		\item $A \lor \lnot A$ (11 аксиома)
		\item $(A \to (A \to A)) \to ((\lnot A \to (A \to A)) \to ((A \lor \lnot A) \to (A \to A)))$ (8 аксиома, подставлены $A, \lnot A, A \to A$)
		\item $(\lnot A \to (A \to A)) \to (A \lor \lnot A) \to (A \to A)$ (modus ponens)
		\item $(A \lor \lnot A) \to (A \to A)$ (modus ponens)
		\item $A \to A$ (modus ponens)
	\end{enumerate}
\end{proof}

\begin{theorem}{(Лемма о дедукции)}
	$\Gamma \cup \{A\} \vdash B \implies \Gamma \vdash (A \to B)$
\end{theorem}

\begin{proof}
	Пусть с набором гипотез $\Gamma \cup \{A\}$ мы могли вывести формулу $B$, последовательно выводя формулы $B_1, B_2, \ldots, B_n, B_n = B$. По индукции докажем, что с набором гипотез $\Gamma$ можно доказать последовательность $A \to B_1, \ldots, A \to B_n$. Разберём для этого все способы, которыми мы умеем выводить
	\begin{enumerate}
	\item $B_i$ получено как гипотеза. Если $B_i = A$, то по лемме мы сможем вывести $A \to A$. Иначе нам доступен такой вывод: $B_i, B_i \to (A \to B_i), A \to B_i$.
	\item $B_i$ получено подставлением формул в аксиому. Работает последовательность из предыдущего пункта.
	\item $B_i$ получено по modus ponens из $B_j$ и $B_k(j < i , k < i)$. Тогда без потери общности считаем, что $B_j = B_k \to B_i$. По предположению индукции мы уже вывели $A \to B_k$ и $A \to (B_k \to B_i)$. По первой аксиоме выведем $A \to (B_k \to B_i) \to ((A \to B_k) \to (A \to B_i))$. Дважды применив к этому modus ponens, получим $A \to B_i$.
	\end{enumerate}
\end{proof}
Некоторые производные правила --- следствия из леммы о дедукции:
\begin{itemize}
	\item Из $A$ и $B$ можно вывести $A \land B$.
	\item Если из $\Gamma \cup \{A\}$ можно вывести $B$ и $\lnot B$, то из $\Gamma$ можно вывести $\lnot A$ (производное правило доказательства от противного).
	\item Если из $\Gamma \cup \{A\}$ можно вывести $B$ и из $\Gamma \cup \{\lnot A\}$ можно вывести $B$, то из $\Gamma$ можно вывести $B$ (производное правило разбора случаев).
	\item Из $A, \lnot A$ можно вывести что угодно.
\end{itemize}

\section{Выразимые (определимые) в данной модели отношения. Теорема о сохранении автоморфизмами выразимых предикатов. Доказательства невыразимости с помощью автоморфизмов.}
\begin{lemma}
	Пусть формула $A$ зависит от переменных $p_1, \ldots, p_n$. При этом при $(p_1, \ldots, p_n) = (\varepsilon_1, \ldots, \varepsilon_n)$ формула выдаёт значение $\varepsilon$. Тогда $\{p_1^{\varepsilon_1}, \ldots, p_n^{\varepsilon_n}\} \vdash A^{\varepsilon}$, где
\[P^\varepsilon = \begin{dcases*} \lnot P, & $\varepsilon = 0$ \\ P& $\varepsilon = 1$ \end{dcases*} \]
\end{lemma}

\begin{proof}
	Доказательство индукцией по построению формулы $A$. Разбираем все способы, которыми она была построена, а для них все значения её составных частей.
	\begin{enumerate}
		\item $A = p$. Очевидно, $p \vdash p$ и $\lnot p \vdash \lnot p$.
		\item $A = B \land C$. 
		\begin{enumerate}
			\item Пусть $B$ и $C$ истинны. Тогда по предположению индукции мы можем вывести $B$ и $C$ и требуется показать, что мы можем вывести $B \land C$. Мы умеем это делать по производному правилу.
			\item Пусть $B$ истинно, а $C$ ложно. Тогда по предположению индукции мы можем вывести $B$ и $\lnot C$, и хотим вывести $\lnot(B \land C)$. Воспользуемся правилом доказательства от противного и добавим себе в гипотезы $B \land C$. Из $B \land C$ нетрудно вывести $C$, и мы умеем выводить $\not C$, противоречие достигнуто.
		\end{enumerate}
		Оставшиеся два случая аналогичны второму.
		\item $A = B \lor C$. Если хотя бы одно из $B$ или $C$ истинно, то ясно, что можно вывести $B \lor C$. Пусть теперь мы умеем выводить $\lnot B, \lnot C$ и нужно вывести $\lnot (B \lor C)$. Снова будем выводить от противного и предположим $B \lor C$. Заметим, что из $\lnot B, \lnot C, B$ можно вывести всё что угодно, и из $\lnot B, \lnot C, C$ можно вывести всё что угодно. Тогда всё что угодно можно вывести и из $\lnot B, \lnot C, B \lor C$, в том числе и противоречие.
		\item $A = B \to C$. Для случаев с истинным $B$ или ложным $C$ вывод простой --- достаточно воспользоваться 2 или 9 аксиомой (и modus ponens). Пусть мы умеем выводить $B, \lnot C$ и нужно вывести $\lnot(B \to C)$. Опять докажем от противного и по modus ponens из $B, B \to C$ выведем $C$. Это даст противоречие, поскольку у нас есть $\lnot C$.
		\item $A = \lnot B$. Ясно, что $\lnot B \vdash \lnot B$. Нужно доказать, что $B \vdash \lnot \lnot B$. Для этого нужно в очередной раз доказать от противного и вывести из $B, \lnot B$ какое-нибудь противоречие. Но $B, \lnot B$ уже противоречие.
	\end{enumerate}
\end{proof}

\begin{theorem}{(Полнота исчисления высказываний)}
	Любую тавтологию можно вывести в исчислении высказываний.
\end{theorem}

\begin{proof}
	Пусть тавтология $A$ зависит от переменных $p_1, \ldots, p_n$. Тогда по лемме $p_1, \ldots, p_n \vdash A$. И $p_1, \ldots, \lnot p_n \vdash A$. И вообще как угодно можно расставить отрицания, потому что $A$ --- тавтология. Из двух приведённых фактов по производному правилу $p_1, \ldots, p_{n - 1}, p_n \lor \lnot p_n \vdash A$. Но $p_n \lor \lnot p_n$ можно получить из аксиомы, значит это можно выкинуть из списка гипотез и получить $p_1, \ldots, p_{n - 1} \vdash A$. Аналогично начав с $p_1, \ldots, \lnot p_{n - 1}, p_n \vdash A$ и $p_1, \ldots, \lnot p_{n - 1}, \lnot p_n \vdash A$, мы получим $p_1, \ldots, \lnot p_{n - 1} \vdash A$. Из этих двух результатов мы сможем избавиться от $p_{n - 1}$ и получить $p_1, \ldots, p_{n - 2} \vdash A$. Долго повторяя этот процесс, мы избавимся от всех переменных и получим $\ \vdash A$, а это то, что требовалось.  
\end{proof}

\section{Нормальные модели. Аксиомы равенства. Теорема о существовании нормальных моделей у непротиворечивых теорий, содержащих аксиомы равенства.}
Если исчисление высказываний работало с произвольными формулами, построенными с помощью отрицания, конъюнкции, дизъюнкции и импликации, исчисление резолюций работает только с дизъюнктами.

\begin{definition} \textit{Литерал} --- переменная или отрицание переменной \end{definition}
\begin{definition} \textit{Дизъюнкт} --- это дизъюнкция по некоторому конечному множеству литералов \end{definition}

Обратите внимание, что в этом определении речь про множество. Хотя мы записываем дизъюнкты как формулы $\lambda_1 \lor \lambda_2 \lor \ldots \lor \lambda_n$, мы считаем, что, к примеру, $\lambda_1 \lor \lambda_2, \lambda_2 \lor \lambda_1, \lambda_1 \lor \lambda_2 \lor \lambda_1$ --- это всё один и тот же дизъюнкт.

У исчисления резолюций нет аксиом и есть одно правило --- правило резолюции
\[\tfrac{A \lor p, \; B \lor \lnot p}{A \lor B} \]
Отметим, что при применении правила к $p$ и $\lnot p$ результатом будет пустой дизъюнкт, который обозначается как $\perp$(или как $\square$). 

На записанные в КНФ пропозиональные формулы можно смотреть как на множества дизъюнктов в исчислении резолюций. Будем говорить, что множество дизъюнктов совместно, если есть набор значений переменных, при котором каждый дизъюнкт возвращает истину. Утверждается, что из множества дизъюнктов можно вывести в исчислени пустой дизъюнкт, если и только если множество несовместно.

\begin{theorem}{(Корректность исчисления резолюций)}
	Если множества дизъюнктов можно вывести пустой дизъюнкт, то оно несовместно.
\end{theorem}
\begin{proof}
	Можно убедиться, что из истинных (при каких-то значениях переменных) формул можно вывести только истинные (при тех же значениях). Но пустой дизъюнкт всегда ложен.

	Если вам по каким-либо причинам не нравятся слова о ложности пустого дизъюнкта, можно сказать, что пустой дизъюнкт можно вывести только из $p, \lnot p$, а они не могут быть истинны одновременно.
\end{proof}
\end{document}