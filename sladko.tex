\documentclass{article}
\usepackage{packages}

\title{Коллоквиум по Дискретной математике, 2 курс}
\author{Залялов Александр, @bcategorytheory,\\  Солодовников Никита, @applied\_memes \\ Шморгунов Александр, @Owlus }
\date{}

\begin{document}

    \maketitle

    \tableofcontents

    \clearpage

    \section{Определение вычислимой частичной функции из $\mathbb{N}$ в $\mathbb{N}$.
    Счётность семейства частичных вычислимых функций, и существование невычислимых функций.
    Разрешимые и перечислимые подмножества $\mathbb{N}$. Счётность семейства перечислимых множеств,
    существование неперечислимых множеств.}

    Здесь не даётся формального определения алгоритма. В нашем случае ``алгоритм'' --- некоторый
    чёрный ящик, принимающий на вход конструктивный объект (натуральное число или же объект, который
    можно закодировать как натуральное число), производящий некоторый результат (также конструктивный
    объект), а также работающий по шагам (некоторые атомарные действия вроде сложения).

    Rule of thumb для вопроса ``Алгоритм ли это?'' --- можно написать как программу на каком-нибудь
    языке программирования.

    \begin{definition}
        Функция $f: \mathbb{N} \to \mathbb{N}$ называется \textit{частичной}, если $\operatorname{Dom} f \subseteq \mathbb{N}$.
    \end{definition}

    \begin{definition}
        Функция $f: \mathbb{N} \to \mathbb{N}$ называется \textit{тотальной}, если $\operatorname{Dom} f = \mathbb{N}.$
    \end{definition}

    \begin{definition}
        Алгоритм $\mathcal{A}$ \textit{вычисляет} частичную функцию $f: \mathbb{N} \to \mathbb{N}$, если
        $$
            \begin{cases}
                \mathcal{A}(x) = f(x), & \text{если $x \in \operatorname{Dom} f$,}\\
                \mathcal{A}(x) \text{ не определено} & \text{иначе.}
            \end{cases}
        $$
    \end{definition}

    \begin{definition}
        Частичная функция $f: \mathbb{N} \to \mathbb{N}$ называется \textit{вычислимой}, если существует
        алгоритм, её вычисляющий.
    \end{definition}

    \begin{statement}
        Множество частичных вычислимых функций не более, чем счётно.
    \end{statement}
    \begin{proof}
        Действительно, всякой вычислимой функции можно поставить в соответствие некоторый алгоритм,
        причём различные функции вычисляются различными алгоритмами. Алгоритм --- это программа,
        то есть конечная строка. Конечных строк (а следовательно, алгоритмов) всего лишь счётное число.
        Существует инъекция из множества вычислимых функций в множество алгоритмов, значит, количество
        вычислимых функций не более, чем счётно.
    \end{proof}

    \begin{theorem}
        Существуют невычислимые функции $f: \mathbb{N} \to \mathbb{N}$.
    \end{theorem}
    \begin{proof}
        Мощность множества вычислимых функций меньше мощности множества всех функций из $\mathbb{N}
        \to \mathbb{N}$, а значит, его дополнение не пусто.
    \end{proof}

    \begin{definition}
        Множество $A \subseteq \mathbb{N}$ называется \textit{разрешимым}, если существует алгоритм, вычисляющий
        его характеристическую функцию $\chi_A(x)$, то есть функцию такую, что
        $$
            \chi_A(x) =
            \begin{cases}
                1, & \text{если $x \in A$,}\\
                0 & \text{иначе.}
            \end{cases}
        $$
    \end{definition}

    \begin{definition}
        Множество $A \subseteq \mathbb{N}$ называется \textit{перечислимым}, если существует
        алгоритм (не принимающий никаких входных данных), который выводит последовательность $a_n$
        такую, что множество всех элементов этой последовательности равно $A$.
    \end{definition}

    \begin{statement}
        Множество перечислимых подмножеств $\mathbb{N}$ не более, чем счётно.
    \end{statement}
    \begin{proof}
        Всякому перечислимому множеству соответствует алгоритм, его перечисляющий, причём различные множества
        перечисляются различными алгоритмами. Отсюда следует, что мощность множества перечислимых множеств не
        превосходит мощности множества алгоритмов, которое, в свою очередь, является счётным.
    \end{proof}

    \begin{theorem}
        Существуют неперечислимые множества $A \subseteq \mathbb{N}$.
    \end{theorem}
    \begin{proof}
        Множество перечислимых множеств имеет мощность меньшую, чем $2^\mathbb{N}$. Значит, его дополнение
        не пусто.
    \end{proof}

    \section{Эквивалентные определения перечислимости: полуразрешимость, область определения
    вычислимой функции, множество значений вычислимой функции.}

    \begin{definition}
        Множество $A \subseteq \mathbb{N}$ называется \textit{полуразрешимым}, если существует алгоритм,
        вычисляющий его полухарактеристическую функцию $\xi_A(x)$, то есть функцию такую, что
        $$
            \xi_A(x) =
            \begin{cases}
                1, & \text{если $x \in A$,}\\
                \text{не определено} & \text{иначе.}
            \end{cases}
        $$
    \end{definition}

    \begin{theorem}
        Следующие утверждения эквивалентны:
        \begin{enumerate}
            \item Множество $A$ перечислимо.
            \item Множество $A$ полуразрешимо.
            \item Существует частичная вычислимая функция $f: \mathbb{N} \to \mathbb{N}$ такая, что
                $A = \operatorname{Dom} f$.
            \item Существует частичная вычислимая функция $f: \mathbb{N} \to \mathbb{N}$ такая, что
                $A = \operatorname{Ran} f$.
        \end{enumerate}
    \end{theorem}

    \begin{proof}
        Чтобы показать эквивалентность всех этих утверждений, докажем несколько импликаций.

        $(1) \implies (2)$

        Множество $A$ перечислимо, докажем его полуразрешимость.

        Модифицируем алгоритм $\mathcal{A}$ перечисления множества $A$ следующим образом: если для входа $x$
        при перечислении мы встретили $x$, вернём ответ $1$, иначе продолжим работу, ничего не возвращая.
        Так как в последовательности, получаемой алгоритмом, рано или поздно встретится каждый из
        элементов $A$, положительный ответ будет дан за конечное число шагов. В случае, если $x
        \not\in A$, алгоритм зациклится без вывода, что вполне устраивает нас в рамках нашей задачи.

        $(2) \implies (3)$

        Множество $A$ полуразрешимо, докажем, что найдётся вычислимая функция, для которой $A$ ---
        область значений.

        Этой частичной вычислимой функцией был \sout{Альберт Эйнштейн} $\xi_A(x)$. Действительно,
        знаем, что $\xi_A(x)$ вычислима, а $\operatorname{Dom} \xi_A = A$. Значит, мы
        нашли искомую функцию.

        $(3) \implies (4)$

        Множество $A$ является областью определения некоторой вычислимой функции $f$, докажем, что
        оно также является областью значений некоторой другой вычислимой функции.

        Определим функцию $g(x)$:
        $$
            g(x) =
            \begin{cases}
                x, & \text{если } x \in \operatorname{Dom} f,\\
                \text{не определено} & \text{иначе.}
            \end{cases}
        $$

        Эта функция вычислима. Алгоритм, её вычисляющий, должен попытаться вычислить $f(x)$ и затем
        просто вывести $x$. Кроме того, $\operatorname{Ran} g = \operatorname{Dom} f$, а значит, мы
        нашли искомую функцию.

        $(4) \implies (1)$

        Множество $A$ является областью значений некоторой вычислимой функции, докажем его
        перечислимость.

        Известно, что существует алгоритм $\mathcal{F}$, вычисляющий функцию $f$, область значений
        которой совпадает с $A$. Чтобы перечислить элементы множества $A$, будем бесконечно
        производить итерации следующего вида: на $n$-той итерации запустим по очереди на $n$
        шагов $\mathcal{F}(i)$ для каждого $0 \leqslant i \leqslant n$. Таким образом, для всех $i$
        алгоритм $\mathcal{F}(i)$ будет рано или поздно запущен на число шагов, необходимое для
        завершения.  Значит, на всех $x$ из $\operatorname{Dom} f$ мы вычислим (и выведем) $f(x)$.
        Таким образом будут выведены все элементы $\operatorname{Ran} f$, равного $A$.

        Из каждого утверждения следуют все остальные. Значит, утверждения эквивалентны.
    \end{proof}

    \section{Теорема Поста. Теорема о графике.}

    \paragraph{Теорема Поста.} \label{Post} Множества $A$ и $\mathbb{N} \setminus A$ перечислимы
    тогда и только тогда, когда $A$ разрешимо.

    \begin{proof} \

        $\implies$

        Перечислимость множества эквивалентна его полуразрешимости. Будем использовать алгоритмы
        $\mathcal{A}$ и $\mathcal{B}$, вычисляющие $\xi_{A}(x)$ и $\xi_{\mathbb{N} \setminus A}(x)$
        соответственно.

        Алгоритм $\mathcal{C}$, находящий $\chi_A(x)$, будет по очереди запускать $\mathcal{A}(x)$ и
        $\mathcal{B}(x)$ на некоторое число шагов. Как только один из этих алгоритмов завершится, можно
        будет дать ответ: если $x \in A$, вернуть $1$, в противном случае вернуть $0$.

        Докажем корректность построенного алгоритма. Во-первых, он действительно даёт правильный
        ответ на всех $x \in \mathbb{N}$, а во-вторых, всегда завершается, так как всякое
        натуральное число лежит либо в множестве $A$, либо в его дополнении. Оба вспомогательных
        алгоритма могут при необходимости отработать бесконечное число шагов, значит, если какой-то
        из них завершается на данном входе, он завершится.

        $\impliedby$

        Если множество разрешимо, его дополнение также разрешимо. Разрешимость влечёт
        перечислимость.
    \end{proof}

    \begin{definition}
        Пусть задана функция $f$. Множество $\Gamma_f = \{(x, f(x)) \mid x \in \operatorname{Dom} f\}$
        называется \textit{графиком} функции $f$.
    \end{definition}

    \paragraph{Теорема о графике.} Функция $f$ вычислима тогда и только тогда, когда $\Gamma_f$
    перечислимо.
    \begin{proof}\

        $\implies$

        Умея вычислять функцию $f$, хотим перечислить $\Gamma_f$.

        Будем бесконечно производить итерации следующего вида: на $n$-той итерации попытаемся
        вычислить $f(x)$ для всех $0 \leqslant x \leqslant n$ не более, чем за $n$ шагов.
        Если удастся, выведем пару $(x, f(x))$ в противном случае остановим вычисление $f(x)$
        и перейдём к следующему $i$. Для каждого $x \in \operatorname{Dom} f$ рано или поздно мы
        произведём достаточное число шагов, чтобы вычислить $f(x)$, так как алгоритм, вычисляющий
        $f(x)$, должен завершаться за конечное число шагов. Значит, $\Gamma_f$ таким образом
        действительно будет перечислено.

        $\impliedby$

        Умея перечислять $\Gamma_f$, хотим вычислить $f(x)$.

        Будем перечислять $\Gamma_f$, пока не найдём пару, в которой первый элемент равен $x$.
        Действительно, если функция определена на $x$, то такая пара найдётся в $\Gamma_f$,
        а значит, будет выведена алгоритмом его перечисления за конечное число шагов. Далее просто
        выведем второй элемент этой пары и завершим работу.

    \end{proof}

    \section{Универсальные вычислимые функции (нумерации) для семейства частичных вычислимых
    функций натурального аргумента. Несуществование универсальной вычислимой функции для
    семейства тотальных вычислимых функций натурального аргумента (диагональное рассуждение).
    Главные универсальные функции.}

    Известно, что множество частичных вычислимых функций счётно. Значит, все эти функции можно
    каким-то способом занумеровать.

    \begin{definition}
        Пусть $\varphi_n$ --- последовательность вычислимых частичных функций. Такая
        последовательность называется \textit{универсальной нумерацией}. Функция $f:
        \mathbb{N} \times \mathbb{N} \to \mathbb{N}$ такая, что $f(n, x) = \varphi_n(x)$, называется
        \textit{универсальной функцией}.
    \end{definition}

    \begin{theorem}
        Существует \underline{вычислимая} нумерация (универсальная функция).
    \end{theorem}

    \begin{proof}
        Определим следующий порядок на двоичных словах: если слово $a$ короче слова $b$, $a \prec
        b$, если наоборот --- $b \prec a$, в случае же равной длины будем сравнивать слова
        лексикографически. Последовательность двоичных слов в таком порядке будет выглядеть как
        $\{\bot, 0, 1, 00, 01, 10, 11, \ldots\}$. Таким образом множество окажется вполне упорядоченным.
        Теперь каждому натуральному числу можно поставить в соответствие некоторое двоичное слово.

        Без ограничения общности будем считать, что алгоритм можно записать некоторым двоичным
        словом.

        Определим теперь функцию $f(n, x)$ следующим образом: интерпретируем двоичное слово
        с номером $n$ (в нашем порядке $\prec$) как код (запись машины Тьюринга, программу на C, whatever)
        для алгоритма $\mathcal{A}$, и положим $f(n, x) = \mathcal{A}(x)$. Если двоичное слово
        с номером $n$ не является корректной записью алгоритма, будем считать, что $f(n, x)$ не
        определена для всех $x$.

        Описанная функция вычислима. Она также является универсальной, так как пробегает по всем
        возможным алгоритмам (и, как следствие, всем возможным вычислимым функциям). Значит, мы
        построили вычислимую универсальную функцию.
    \end{proof}

    Может показаться, что универсальная функция может существовать и для семейства тотальных
    вычислимых функций. Однако, это неверно.

    \begin{theorem}
        Не существует вычислимой нумерации (универсальной функции) для семейства тотальных
        вычислимых функций.
    \end{theorem}
    \begin{proof}
        Допустим, что существует вычислимая нумерация $\psi_n$ тотальных вычислимых функций. Значит,
        будет тотальной и вычислимой функция следующего вида
        $$
            f(x) = \psi_x(x) + 1.
        $$

        Так как функция $f$ тотальна и вычислима, должно найтись $n$ такое, что $\psi_n(x) = f(x)$.
        Однако, если поставить $x = n$:
        $$
            \psi_n(n) = f(n) = \psi_n(n) + 1.
        $$

        Произошло противоречие \footnote{Такая конструкция не будет приводить к противоречию,
        если говорить о частичных функциях, а не тотальных. Действительно, построенная нами
        функция $f = \varphi_n$ просто не будет определена в точке $n$.}\dots

    \end{proof}

    \begin{definition}
        Нумерация $\varphi$ называется \textit{главной}, если для любой вычислимой частичной
        функции $V(n, x)$ существует тотальная вычислимая функция $s(n)$ такая, что $V(n, x)
        = \varphi_{s(n)}(x)$.
    \end{definition}

    \begin{theorem}
        Существует главная нумерация (универсальная функция).
    \end{theorem}

    \begin{proof}
        Рассмотрим построенную выше нумерацию. Для функции $V(n, x)$ построим алгоритм $\mathcal{S}(m)$,
        который ``вшивает'' в алгоритм $\mathcal{V}(n, x)$, вычисляющий $V(n, x)$, константу $m$ вместо
        переменной $n$. Такой алгоритм, очевидно, всегда будет завершаться. Значит, он вычисляет
        некоторую тотальную функцию $s(n)$.
    \end{proof}

    \section{Вычислимая функция, не имеющая тотального вычислимого продолжения. Перечислимое неразрешимое
    множество. Неразрешимость \\ проблемы применимости.}

    \begin{definition}
        Функция $g$ является \textit{продолжением} функции $f$, если $\operatorname{Dom} f \subset
        \operatorname{Dom} g$ и $\forall x \in \operatorname{Dom} f \; g(x) = f(x)$.
    \end{definition}

    \begin{theorem}
        Существует вычислимая частичная функция, не имеющая всюду определённого продолжения.
    \end{theorem}
    \begin{proof}
        Определим функцию $f$:
        $$
            f(x) = \varphi_x(x) + 1.
        $$

        Пусть мы вычислимо продолжили функцию $f$ функцией $\varphi_n$ (продолжение вычислимо, а потому
        будет присутствовать в главной нумерации). Запишем уравнение:
        $$
            \varphi_x(x) + 1 = \varphi_n(x).
        $$

        Это уравнение, вообще говоря, неверно, так как левая часть может быть не определена, но при
        этом при подстановке $x = n$ получаем слева вполне определённое выражение (напомню,
        $\varphi_n$ всюду определена), а вместе с ним и противоречие:
        $$
            \varphi_n(n) + 1 = \varphi_n(n).
        $$

        Значит, для функции $f$ не существует всюду определённого вычислимого продолжения.
    \end{proof}

    \begin{theorem}
        Множество $A = \{x \mid \varphi_x(x) \text{ определено}\}$ неразрешимо.
    \end{theorem}
    \begin{proof}
        Вернёмся вновь к нашей любимой функции $f$:
        $$
            f(x) = \varphi_x(x) + 1.
        $$

        Продолжим эту функцию самым простым способом --- везде вне её области определения её
        продолжение будет равно нулю. Нетрудно было бы реализовать это продолжение, умея разрешать
        множество $A$: просто вычислим $\varphi_x(x) + 1$, если это значение определено, и вернём ноль
        иначе.

        Однако мы уже знаем, что продолжение функции $f$ не может быть вычислимым. Значит,
        вычислимость $\chi_A$ приводит к противоречию.
    \end{proof}

    \begin{definition}
        Задача разрешения множества $\{(n, x) \mid \varphi_n(x) \text{ определено}\}$ называется
        \textit{проблемой остановки (применимости)}.
    \end{definition}

    \begin{theorem}
        \textit{Проблема остановки} неразрешима.
    \end{theorem}
    \begin{proof}
        Пусть $\chi_A$ вычисляется алгоритмом $\mathcal{A}$. Запустив $\mathcal{A}(x, x)$, можно
        разрешить множество $\linebreak{\{x \mid \varphi_x(x) \text{ определено}\}}$, для которого уже
        доказана неразрешимость.
    \end{proof}

    \section{Теорема Поста. Существование перечислимого множества, дополнение которого неперечислимо.
    Перечислимые неотделимые множества.}

    \textbf{Теорема Поста} была доказана \hyperref[Post]{здесь}.

    \begin{theorem}
        Существует перечислимое множество с неперечислимым дополнением.
    \end{theorem}
    \begin{proof}
        Уже знакомое нам множество $\{x \mid \varphi_x(x) \text{ определено}\}$ является
        неразрешимым. Однако же, это множество очевидно полуразрешимо, а значит, и перечислимо.

        Предположим теперь, что дополнение данного множества перечислимо. В таком случае, согласно
        теореме Поста, само множество разрешимо. Противоречие.
    \end{proof}

    \begin{definition}
        Множества $A, B$ называются \textit{отделимыми}, если существует множество $C$ такое, что $A
        \subseteq C$ и $B \cap C = \varnothing$.
    \end{definition}

    \begin{theorem}
        Существуют непересекающиеся перечислимые множества, которые нельзя отделить разрешимым
        множеством.
    \end{theorem}
    \begin{proof}
        Рассмотрим множества $A = \{x \mid \varphi_x(x) = 0\}$ и $B = \{x \mid \varphi_x(x) = 42\}$.
        Они, очевидно, перечислимы и не пересекаются.

        Допустим, существует разрешимое $C$, отделяющее $A$ и $B$. Пусть оно содержит в себе
        множество $A$. Тогда:

        $$
            \chi_C(x) =
            \begin{cases}
                1, & \text{если $\varphi_x(x) = 0$,}\\
                0, & \text{если $\varphi_x(x) = 42$,}\\
                \text{и что-то ещё на других числах.}
            \end{cases}
        $$

        $\chi_C$ вычислима. Значит, $\exists n \ \chi_C(x) = \varphi_n(x)$ Но пусть тогда $x = n$.
        Получаем:

        $$
            \varphi_n(n) =
            \begin{cases}
                1, & \text{если $\varphi_n(n) = 0$,}\\
                0, & \text{если $\varphi_x(n) = 42$,}\\
                \text{и что-то ещё на других числах.}
            \end{cases}
        $$

        Получили противоречие.
    \end{proof}

    \section{Сводимости: m-сводимость и Тьюрингова сводимость. Свойства. Полные перечислимые множества.}

    \begin{definition}
        Множество $A$ \textit{m-сводится} к множеству $B$, если существует тотальная вычислимая
        функция $f$ такая, что $\forall x \ x \in A \iff f(x) \in B$. Обозначается как $A
        \leqslant_m B$.
    \end{definition}

    m-сводимость позволяет построить алгоритм разрешения множества $A$, если есть алгоритм
    разрешения множества $B$. Строго говоря, $\chi_A(x) = \chi_B(f(x))$.

    Свойства m-сводимости:
    \begin{itemize}
        \item $A \leqslant_m A$ (рефлексивность),
        \item $A \leqslant_m B \wedge B \leqslant_m C \implies A \leqslant_m C$ (транзитивность),
        \item
        $
        \left.\begin{aligned}
                &B \text{ разрешимо}\\
                &A \leqslant_m B
        \end{aligned}\right\}
        \implies A \text{ разрешимо},
        $
        \item
        $
        \left.\begin{aligned}
                &A \text{ неразрешимо}\\
                &A \leqslant_m B
        \end{aligned}\right\}
        \implies B \text{ неразрешимо}.
        $
    \end{itemize}

    \begin{definition}
        Множество $A$ \textit{T-сводится} (сводится по Тьюрингу) к множеству $B$, если при помощи
        алгоритма вычисления $\chi_B$ (не обязательно существующего) можно вычислить $\chi_A$.
        Обозначается как $A \leqslant_T B$ \footnote{Это --- старая добрая сводимость из курса
        алгоритмов. Классы задач P и NP машут ручкой.}.
    \end{definition}

    Тьюринова сводимость обладает теми же свойствами, что и m-сводимость. Однако же, $A \leqslant_m
    B \implies A \leqslant_T B$, а обратное утверждение неверно.

    Тьюрингова сводимость обладает ещё одним свойством: $A \leqslant_T \mathbb{N} \setminus A$.
    Однако же данное утверждение не будет верно для m-сводимости. К примеру, множество $\mathbb{N}$
    не может быть m-сведено к своему дополнению.

    \begin{definition}
        Перечислимое множество, к которому m-сводится любое другое перечислимое множество, называется
        \textit{полным перечислимым множеством}.
    \end{definition}
    \begin{theorem}
        Существует полное перечислимое множество.
    \end{theorem}
    \begin{proof}
        Рассмотрим множество $A = \{(n, x) \mid \varphi_n(x) \text{ определено}\}$. Заметим, что оно
        перечислимо.

        Пусть мы хотим свести некоторое перечислимое множество $B$ к множеству $A$. Знаем, что
        в силу перечислимости существует вычислимая частичная функция $f$ такая, что $B
        = \operatorname{Dom} f$. Эта функция должна присутствовать в универсальной нумерации. Пусть
        это $\varphi_n$. Тогда для того, чтобы проверить принадлежность $x \in B$, достаточно проверить
        принадлежность $(n, x) \in A$. Сводящая функция в данном случае выглядит как $m(x) = (n,
        x)$.
    \end{proof}

    \section{Теорема Клини о неподвижной точке.}

    \begin{theorem}
        Для всякой тотальной вычислимой функции $f$ и главной нумерации $\varphi_n$ найдётся $n$
        такое, что $\varphi_n = \varphi_{f(n)}$ \footnote{Неформально: существует программа, которая
        может напечатать свой код.}.
    \end{theorem}

    \begin{proof}
        Педагогический трюк для лучшего запоминания: сначала попытаемся доказать ложное утверждение
        о том, что у всякой тотальной вычислимой функции есть неподвижная точка, то есть число $n$
        такое, что $n = f(n)$.

        Функция $f$ вычислима, функция $\varphi_x(x)$ вычислима, значит, вычислима их композиция. То
        есть, существует $m$ такое, что
        $$
            f(\varphi_x(x)) = \varphi_m(x).
        $$

        Подставляя $x = m$, получаем
        $$
            f(\varphi_m(m)) = \varphi_m(m),
        $$
        то есть $n = \varphi_m(m)$.

        И всё бы было хорошо, если бы $\varphi_m(m)$ было всегда определено. Но вернёмся в суровую
        реальность и докажем истинное утверждение теоремы Клини.

        Вместо равенства чисел нужно рассматривать эквивалентность следующего вида: $n \sim m$,
        если $\varphi_n = \varphi_m$.

        Рассмотрим вычислимую функцию $$V(m, x) = \varphi_{f(\varphi_m(m))}(x).$$ По свойству главности
        нумерации $\varphi$ найдётся тотальная вычислимая функция $s$ такая, что $$V(m, x)
        = \varphi_{s(m)}(x).$$ Значит, можем записать, что
        $$
            f(\varphi_m(m)) \sim s(m).
        $$

        Важно отметить, что $s(m)$ тотальна. Какое бы значение $m$ мы не подставили, наше
        утверждение сохранит истинность в силу того, что правая часть уравнения определена.

        Теперь, зная, что $s(m)$ вычислима, запишем её как $\varphi_k(m)$
        $$
            f(\varphi_m(m)) \sim \varphi_k(m),
        $$
        и подставим $m = k$
        $$
            f(\varphi_k(k)) \sim \varphi_k(k).
        $$

        Чудесным образом получили слева и справа вполне определённые значения, так как и $f$,
        и $\varphi_k$ являются тотальными функциями. То есть, неподвижной точкой (в смысле
        определённой нами эквивалентности, а не обычного равенства) функции $f$ является
        $\varphi_k(k)$, совершенно точно определённое значение.
    \end{proof}

    \section{Теорема Райса-Успенского.}

    \begin{theorem}
        Пусть множество $F$ является собственным подмножеством множества частичных вычислимых
        функций (то есть, $F$ не пусто и не совпадает с множеством всех частичных вычислимых
        функций). Тогда множество $\{i \mid \varphi_i \in F\}$ неразрешимо \footnote{Неформально
        эту теорему можно понимать так: по алгоритму вычисления функции нельзя понять
        (алгоритмически), обладает ли она каким-либо свойством. То есть, к примеру, нельзя
        написать алгоритм, решающий, монотонна ли функция, по коду, её вычисляющему.}.
    \end{theorem}

    \begin{proof}
        Обозначим нигде не определённую функцию как $\zeta(x)$. Рассмотрим два случая.

        \begin{itemize}
            \item $\zeta \not\in F, f \in F$

            Пусть $A$ --- перечислимое неразрешимое множество. Хотим m-свести $A$ к $F$. Для этого
            определим функцию $V(n, x)$ следующим образом:
            $$
                V(n, x) =
                \begin{cases}
                    f(x), & \text{$n \in A$,}\\
                    \zeta(x), & \text{иначе.}
                \end{cases}
            $$

            Так как полухарактеристическая функция $A$ вычислима, $V(n, x)$ также вычислима. Кроме
            того, в главной нумерации $\varphi$ найдётся тотальная вычислимая функция $s$ такая, что
            $V(n, x) = \varphi_{s(n)}(x)$. Запишем:
            $$
                \varphi_{s(n)}(x) =
                \begin{cases}
                    f(x), & \text{$n \in A$,}\\
                    \zeta(x), & \text{иначе.}
                \end{cases}
            $$

            Но теперь мы получили, что $n \in A \iff s(n) \in F$. Значит, неразрешимое множество $A$
            m-сводится к нашему множеству $F$. Значит, и $F$ является неразрешимым.

            \item $\zeta \in F, f \not\in F$

            Построив всё аналогично предыдущему пункту, мы получаем, что $n \in A \iff s(n) \not\in F$.
            Это значит, что дополнение $F$ неразрешимо. Само $F$ в таком случае также неразрешимо.
        \end{itemize}
    \end{proof}

\end{document}
